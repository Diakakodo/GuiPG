\section{Les risques survenus}
  \subsection{Risques organisationnels}
    \begin{frame}
      \frametitle{\color{white}Risques organisationnels}
      \begin{block}{Perte de membres}
        Deux membres ont quitter le projet durant le premier sprint.
      \end{block}
      \begin{block}{Incompatibilité d'emploi du temps}
        Deux membres sont passé en GIL.\\
        Difficultés pour se réunir.
      \end{block}
      \pause
      \begin{exampleblock}{Solutions}
        \begin{itemize}
          \item Mis en place d'un plan d'action
          \item Redéfinition du périmètre du projet
          \item Mise en place d'un créneau hebdomadaire commun.
        \end{itemize}
      \end{exampleblock}
    \end{frame}

  \subsection{Risques techniques}
    \begin{frame}
      \frametitle{\color{white}Maîtrise des outils}
      \begin{block}{Maîtrise des outils}
        Majorité des membres pas assez formé sur les outils utilisé\\
        (git, Qt, C++, gpg)
      \end{block}
      \pause
      \begin{exampleblock}{Solutions}
        \begin{itemize}
          \item Réunion d'installation et de formation aux outils
        \end{itemize}
      \end{exampleblock}
    \end{frame}
    \begin{frame}
      \frametitle{\color{white}Fonctionnement technique de GnuPG}
      \begin{block}{Fonctionnement technique de GnuPG}
        GnuPG utilise directement un tty comme entrée / sortie pour certaine commandes.\\
        Ce qui nous a compliqué la tâche pour arriver a envoyer
        et récupérer de donnée du logiciel.
      \end{block}
      \pause
      \begin{exampleblock}{Solutions}
        \begin{itemize}
          \item Inspection du code source de GnuPG et de son comportement
          \item Réalisation d'un pseudo-terminal
        \end{itemize}
      \end{exampleblock}
    \end{frame}

\section{Bilan du projet}
  \begin{frame}
    \frametitle{\color{white}Bilan du projet}
  \end{frame}
