\section{PGP}
\subsection{PGP}
\begin{frame}
    \frametitle{\color{white}PGP}
    \begin{block}{Qu'est-ce que PGP ?}
      \begin{itemize}
       \item Logiciel créé par Phil Zimmermann en 1991.
         \item Pretty Good Privacy = Assez bonne confidentialité
       \end{itemize} 
    \end{block}
    \begin{block}{Pourquoi ?}
      \begin{itemize}
         \item Permettre l'utilisation de la cryptographie pour le grand public
         \item Chiffrer, déchiffrer, signer et vérifier des données. 
         \item Échanger des emails de manière sûre.
       \end{itemize} 
    \end{block}
\end{frame}
\begin{frame}
    \frametitle{\color{white}PGP}
    \begin{block}{Comment ?}
    	\begin{itemize}
         \item Un utilisateur doit donc posséder:
	  \begin{itemize}
	    \item Sa propre paire de clefs asymétriques (privée et public).
	    \item Les clefs publiques des personnes avec qui il souhaite communiquer. 
	  \end{itemize}
	 \item Une clef publique sert à chiffrer et véfifier.
	 \item Une clef privée sert à déchiffrer ou signer.
	 \item Les utilisateurs doivent donc partager leur clef publique avec leurs contacts.
       \end{itemize} 
    \end{block}
\end{frame}

\begin{frame}
    \frametitle{\color{white}PGP - Les Clefs}
    \begin{block}{Qu'est ce qu'une clef PGP ?}
    	\begin{itemize}
	  \item Une clef asymétrique privée ou publique.
	  \item Une date de création.
	  \item Une empreinte ( = le hashé de : la clef + la date de création)
	  \item Un utilisateur (nom et adresse email).
	  \item Une auto-signature (faite par cette même clef).
       \end{itemize} 
    \end{block}
    \begin{block}{Une clef est associé en plus à :}
      \begin{itemize}
        \item Un niveau de validité.
	\item Un niveau de confiance.
      \end{itemize}
    \end{block}

\end{frame}

\begin{frame}
  \frametitle{\color{white}PGP - Le modèle de confiance}
    \begin{block}{Le concept de toile de confiance}
    	\begin{itemize}
	  \item Les amis de mes amis sont mes amis.
	  \item Certification de clef.
	  \item Attribution d'un niveau de confiance.
       \end{itemize} 
    \end{block}
\end{frame}
\begin{frame}
  \frametitle{\color{white}PGP - Le modèle de confiance}
    \begin{block}{Validité non définie}
      Alice n'a pas attribué de confiance aux trois clefs qu'elle a certifiées.\\
      La clé non certifiée par Alice n'est donc pas valide.
    \end{block}
    \includegraphics[scale=0.3]{tdcdemoUndefined.png}
\end{frame}
\begin{frame}
  \frametitle{\color{white}PGP - Le modèle de confiance}
    \begin{block}{Validité Marginal}
      Une ou deux clefs certifiées par Alice avec une confiance légère 
      induisent une validité marginale sur la clé non certifiée par Alice.
    \end{block}
    \includegraphics[scale=0.3]{tdcdemoMarginal.png}
\end{frame}
\begin{frame}
  \frametitle{\color{white}PGP - Le modèle de confiance}
    \begin{block}{Validité Complète}
      Trois clefs ou plus certifiés par Alice avec une confiance légère
      induite une validité complète sur la clé non certifiée par Alice.
    \end{block}
    \includegraphics[scale=0.3]{tdcdemoComplete1.png}
\end{frame}
\begin{frame}
  \frametitle{\color{white}PGP - Le modèle de confiance}
    \begin{block}{Validité Complète}
      Une clef certifiée par Alice avec une confiance complète
      suffit à propager une validité complète sur la clé non certifiée par Alice.
    \end{block}
    \includegraphics[scale=0.3]{tdcdemoComplete2.png}
\end{frame}

\subsection{OpenPGP}
\begin{frame}
    \frametitle{\color{white}OpenPGP}
    \begin{block}{Qu'est-ce que OpenPGP ?}
    	\begin{itemize}
    	 \item Crée en 1997 par l'IETF (Internet Engineering Task Force)
    	 \item Protocol libre permettant de sécuriser l'échange d'email.
    	 \item Définit différents formats de paquets (message chiffré, signature, clef privée, clef publique...).
    	 \item Se base sur le logiciel PGP.
    	 \item Normalisé dans la RFC 2440 (novembre 1998).
         \item Normalisé dans la RFC 4880 (novembre 2007).
       \end{itemize} 
    \end{block}
\end{frame}

\subsection{GnuPG}
\begin{frame}
    \frametitle{\color{white}GnuPG}
    \begin{block}{Qu'est-ce que GnuPG ?}
      \begin{itemize}
        \item GNU Privacy Guard (GPG)
        \item Logiciel libre créé par Werner KOCH en 1997.
        \item Basé sur le standard OpenPGP.
        \item Alternative à PGP.
      \end{itemize}
    \end{block}
\end{frame}