\documentclass[a4paper,11pt,french]{article}

%Import des packages utilisés pour le document
\usepackage[utf8]{inputenc}
\usepackage[french]{babel}
\usepackage{chngpage}
\usepackage{hyperref}
\usepackage{fontspec,xltxtra,xunicode,color}
\usepackage{tabularx, longtable}
\usepackage[table]{xcolor}
\usepackage{fancyhdr}
\usepackage{lastpage}

\definecolor{gris}{rgb}{0.95, 0.95, 0.95}

%Redéfinition des marges
\addtolength{\hoffset}{-2cm}
\addtolength{\textwidth}{4cm}
\addtolength{\topmargin}{-2cm}
\addtolength{\textheight}{1cm}
\addtolength{\headsep}{0.8cm} 
\addtolength{\footskip}{1cm}


%Import page de garde et structures pour la gestion de projet
\usepackage{res/structures
}

%Variables
\def\matiere{Conduite de Projet}
\def\filiere{Master 1 SSI}
\def\projectDesc{Interface graphique GPG}
\def\projectName{\emph{GPG}~}
\def\completeName{\projectDesc ~- \projectName}
\def\docType{Spécification Technique du Besoin}
\def\docDate{\today}
\def\version{1.0}
\def\author{Matthieu \textsc{Fin}}
\def\checked{Olivier \textsc{Thibault}}
\def\approved{Magalie \textsc{Bardet}}



% -- Début du document -- %
\begin{document}
%Page de garde
\makeFirstPage
\clearpage

%Tableau de mises à jour
\vspace*{1cm}
\begin{center}
\textbf{\huge{MISES À JOUR}}\\
\vspace*{3cm}
	\begin{tabularx}{16cm}{|c|c|X|}
	\hline
	\bfseries{Version} & \bfseries{Date} & \bfseries{Modifications réalisées}\\
	\hline
	0.1 & 30/10/2014 & Création\\
	\hline
	 & &\\
	\hline
	 & & \\
	\hline
	&&\\
	\hline
	&&\\
	\hline
	\end{tabularx}
\end{center}

%La table des matières
\clearpage
\tableofcontents
\clearpage

\section{Objet}
\textcolor{blue}{
  Présenter succinctement le sujet et les hypothèses de travail:
  \begin{itemize}
  \item Besoin opérationnel et contexte d'exploitation (unité du logiciel)
  \item Ojectifs techniques
  \item Contraintes et recommandations
  \item Résultats attendus
  \end{itemize}
}

\section{Documents applicables et de référence}
\textcolor{blue}{
  Lister
  \begin{itemize}
  \item Les références des documents qui définissent formellement les principes
    directeurs et les hypothèses de travail prise en compte pour l’établissement de la spécification.
  \item Les références des documents cités dans la STB au titre d’explication ou de justification
  \end{itemize}
}

\section{Terminologie et sigles utilisés}
\textcolor{blue}{
  \begin{temize}
  \item Glossaire ou dictionnaire
  \item Abréviations
  \item Formalisme utilisé
  \item Légendes et conventions de représentation
  \end{itemize}
}

\section{Exigences fonctionnelles}
\subsection{Présentation de la mission du produit logiciel}
\textcolor{blue}{
  Exprimer le besoin fonctionnel:
  \begin{itemize}
  \item Décrire l’environnement d’exploitation 
    (acteurs, positionnement dans le système, profils des utilisateurs, etc.)
  \item Recenser les cas d’utilisation et dresser une liste des scénarios de mise en œuvre du système.
  \end{itemize}
  Ce chapitre pourra être illustré par des diagrammes de cas d’utilisation UML.\\
  La liste des cas d’utilisation pourra être présentée dans un tableau 
  (voir exemple  proposé ci-après)
}
\begin{tabular}{|>{\centering}p{1cm}|>{\centering}p{7cm}|>{\centering}p{3.5cm}|>{\centering}p{3cm}|}
  \hline
  \color{white}\cellcolor{blue}\bfseries{Id}&
  \color{white}\cellcolor{blue}\bfseries{Intitulé}&
  \color{white}\cellcolor{blue}\bfseries{Acteur(s)}&
  \color{white}\cellcolor{blue}\bfseries{Priorité}&
  \hline
  1&
  Création d'un nouveau compte&
  Administrateur&
  Indispensable&
  \hline
  2&
  Consultation de statistiques&
  Manager&
  Important&
  \hline
  3&
  Récapitulatif général&
  Manager&
  Secondaire&
  \hline
  &
  &
  &
  &
  \hline
\end{tabular}\\
\textcolor{blue}{pour chaque cas d'utilisation identifié.}
\clearpage

\subsection{Cas d'utilisation C1}

\textcolor{blue}{
  Chaque cas devra faire l’objet d’un paragraphe contenant une description détaillée 
  sous forme de fiche servant de référence pour le projet. Les fiches pourront être 
  complétées et mises à jour pendant le développement en fonction des exigences 
  nouvelles ou des évolutions demandées par le client.
}

\section{Exigences opérationnelles}

\textcolor{blue}{
  Fournir la liste des exigences de mise en œuvre ou de performance et les identifier par un numéro.
}

\section{Exigences opérationnelles d'interface}

\textcolor{blue}{
  Fournir la liste des exigences d’interface avec d’autres logiciels ou avec des matériels 
  spécifiques et les identifier par un numéro.
}

\section{Exigences de qualité}

\textcolor{blue}{
  Fournir la liste des exigences particulières de qualité et les identifier par un numéro.
}

\section{Exigences de réalisation}

\textcolor{blue}{
  Fournir la liste des exigences techniques de réalisation et les identifier par un numéro.
}

\end{document}

