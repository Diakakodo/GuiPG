\documentclass{../res/univ-projet}

%Import des packages utilisés pour le document
\usepackage[utf8x]{inputenc}
\usepackage[francais]{babel}
\usepackage[T1]{fontenc}
%\usepackage{array}
%\usepackage{hyperref}
%\usepackage{tabularx, longtable}
%\usepackage[table]{xcolor}
%\usepackage{fancyhdr}
%\usepackage{lastpage}

\definecolor{gris}{rgb}{0.95, 0.95, 0.95}

%Redéfinition des marges
%\addtolength{\hoffset}{-2cm}
%\addtolength{\textwidth}{4cm}
\addtolength{\topmargin}{-1cm}
\addtolength{\textheight}{1cm}
\addtolength{\headsep}{0.8cm} 
\addtolength{\footskip}{-0.2cm}


%Import page de garde et structures pour la gestion de projet
%\usepackage{structures}

%Variables
\logo{../res/logo_univ.png}
\title{Spécification Technique de Besoin}
\author{Ibrahima-Sory \bsc{Barry}, Bertille \bsc{Bouillie}}
\projet{GPG}
\projdesc{Interface Graphique GPG}
\filiere{M1SSI - Conduite de Projet}
\version{0.1}
\relecteur{Olivier \bsc{Thibault}}
\signataire{Magali \bsc{Bardet}}
\date{\today}

\histentry{0.1}{31/10/2014}{Version initiale.}

% -- Début du document -- %
\begin{document}

%Page de garde
\maketitle
\newpage
%La table des matières
\tableofcontents
\newpage

\section{Objet}

% Présentation succinte du sujet et hyp de travail.
Ce document est la Spécification Technique du Besoin de la réalisation d'une interface graphique pour le logiciel GnuPG.
Réalisation d'une interface graphique permettant d'utiliser complètement l'outil OpenPGP
de façon intuitive et pédagogique.

\subsection{Besoins opérationnels}
\begin{itemize}
 \item L'application doit fonctionner sur le système d'exploitation GNU/Linux.
\end{itemize}

\subsection{Objectifs techniques}
\begin{itemize}
 \item Gestion fine des clés.
 \item Gestion et aperçu de la toile de confiance.
 \item Chiffrement et déchiffrement pour le webmailing.
\end{itemize}

\subsection{Contraintes et recommandations}
\begin{itemize}
 \item L'application doit respecter les normes RFC.
 \item L'application doit être pédagogique et précise.
 \item Chaque action devra être accompagnée d'une visualisation de la ligne de commande GnuPG équivalente.
\end{itemize}

\subsection{Résultats attendus}
\begin{itemize}
 \item 
\end{itemize}



\section{Documents applicables et de référence}
% Liste des
% - Références des documents quidefinissent formellement les principes
%   directeurs et le hypothèse de travail prise en compte pour l'établissement de la spécification.
% - Références des documents cités dans la STB au titre d'explication ou de justification.
Différents documents de référence :
\begin{itemize}
\item Définitions du standard OpenPGP \href{file:../../ressources/openPGP/rfc4880-en.pdf}{RFC 4880}
  et \href{file:../../ressources/openPGP/rfc2440-fr.pdf}{RFC 2440}
\item Le logiciel \href{https://www.gnupg.org/}{GnuPG} (GNU Privacy Guard) implantation Open Source
  de OpenPGP.
\item La \href{https://www.gnupg.org/gph/fr/manual.html#AEN541}{toile de confiance} de GnuPG
\item Editeurs graphiques existant 
  (\href{http://www.gnupg.org/related_software/frontends.en.html}{KGpg, GPA, Seahorse})
\item Exemple de visualisation d'une toile de confiance sur le site de 
  \href{https://www.archlinux.org/master-keys/#visualization}{archlinux}.
\end{itemize}

\section{Terminologie et sigles utilisés}
\textcolor{blue}{
  \begin{itemize}
  \item Glossaire ou dictionnaire
  \item Abréviations
  \item Formalisme utilisé
  \item Légendes et conventions de représentation
  \end{itemize}
}

\section{Exigences fonctionnelles}
\subsection{Présentation de la mission du produit logiciel}

\begin{tabular}{|>{\centering}p{1cm}|>{\centering}p{7cm}|>{\centering}p{2.5cm}|>{\centering}p{3cm}|}
  \hline
  \color{white}\cellcolor{blue}\bfseries{Id}&
  \color{white}\cellcolor{blue}\bfseries{Intitulé}&
  \color{white}\cellcolor{blue}\bfseries{Acteur(s)}&
  \color{white}\cellcolor{blue}\bfseries{Priorité}\\
  \cr
  \hline
  1&
  Actions GPG&
  Utilisateur&
  Indispensable
  \cr
  \hline
  2&
  Exportation d'une nouvelle clé&
  Utilisateur&
  Indispensable
  \cr
  \hline
  3&
  Importation d'une nouvelle clé&
  Utilisateur&
  Indispensable
  \cr
  \hline
  4&
  Génération d'un certificat de révocation&
  Utilisateur&
  Important
  \cr
  \hline
  5&
  Révocation d'une clé&
  Utilisateur&
  Important
  \cr
  \hline
  6&
  Chiffrement ou déchiffrement d'un document&
  Utilisateur&
  Indispensable
  \cr
  \hline
  7&
  Gestion de la toile de confiance&
  Utilisateur&
  Indispensable
  \cr
  \hline
\end{tabular}\\

\newpage

%Cas d'utilisation
\subsection{Cas d'utilisation 1}
\ficheGraphic
{Actions GPG}                                    % Nom du cas d'utilisation
{Utilisateur}                                    % Acteurs concernés
{                                                % Description
  Fonctionnement du système pour chaques actions
  necessitant l'intervention de GPG.
  Toutes ces actions sont spécifiée dans les
  \newline
  \href{file:../../ressources/openPGP/rfc4880-en.pdf}{RFC 4880}
  et \href{file:../../ressources/openPGP/rfc2440-fr.pdf}{RFC 2440}
}
{}                                               % Préconditions
{}                                               % Evénements déclenchants
{Action terminée}                                % Conditions d'arrêt
{0.6}{../res/graphics/Diag_Seq_Actions_GPG.jpeg} % Diagramme
{                                                % Flots d'exceptions
  \begin{itemize}
  \item Abandon de l'utilisateur.
  \item Erreurs GPG.
  \end{itemize}
}
% Fin de la fiche du cas d'utilisation 1.
\vspace{0.5cm}

\subsection{Cas d'utilisation 2}
\fiche{Exportation d’une nouvelle clé}{Utilisateur}{Exportation d’une nouvelle clé}{}{}{Clé exportée}
{1.}
{2.}
{Abandon de l'utilisateur}
\vspace{0.5cm}

\subsection{Cas d'utilisation 3}
\fiche{Importation d’une nouvelle clé}{Utilisateur}{Importation d’une nouvelle clé}{}{}{Clé importée et éventuellement signée}
{1.}
{2.}
{Abandon de l'utilisateur}
\vspace{0.5cm}

\subsection{Cas d'utilisation 4}
\fiche{Génération d’un certificat de révocation}{Utilisateur}{Génération d’un certificat de révocation}{}{}{Certificat de révocation créé}
{1.}
{2.}
{Abandon de l'utilisateur}
\vspace{0.5cm}

\subsection{Cas d'utilisation 5}
\fiche{Révocation d’une clé}{Utilisateur}{Révocation d’une clé}{}{}{Clé révoquée}
{1.}
{2.}
{Abandon de l'utilisateur}
\vspace{0.5cm}

\subsection{Cas d'utilisation 6}
\fiche{Chiffrement ou déchiffrement d’un document}{Utilisateur}{Chiffrement ou déchiffrement d’un document}{}{}{Document chiffré ou déchiffré}
{1.}
{2.}
{Abandon de l'utilisateur}
\vspace{0.5cm}

\subsection{Cas d'utilisation 7}
\fiche{Gestion de la toile de confiance}{Utilisateur}{Gestion de la toile de confiance}{}{}{}
{1.}
{2.}
{Abandon de l'utilisateur}
\vspace{0.5cm}




\section{Exigences opérationnelles}

\begin{tabular}{|>{\centering}p{1,5cm}|>{\centering}p{10cm}|>{\centering}p{3cm}|}
  \hline
  \color{white}\cellcolor{blue}\bfseries{Id}&
  \color{white}\cellcolor{blue}\bfseries{Intitulé}&
  \color{white}\cellcolor{blue}\bfseries{Priorité}\\
  \cr
  \hline
  EO1&
  &
  Indispensable
  \cr
  \hline
\end{tabular}\\


\section{Exigences opérationnelles d'interface}

\begin{tabular}{|>{\centering}p{1,5cm}|>{\centering}p{10cm}|>{\centering}p{3cm}|}
  \hline
  \color{white}\cellcolor{blue}\bfseries{Id}&
  \color{white}\cellcolor{blue}\bfseries{Intitulé}&
  \color{white}\cellcolor{blue}\bfseries{Priorité}\\
  \cr
  \hline
  EI1&
  KDE&
  Indispensable
  \cr
  \hline
  EI2&
  Gnome&
  Indispensable
  \cr
  \hline
\end{tabular}\\


\section{Exigences de qualité}

\textcolor{blue}{
  Fournir la liste des exigences particulières de qualité et les identifier par un numéro.
}

\section{Exigences de réalisation}

\textcolor{blue}{
  Fournir la liste des exigences techniques de réalisation et les identifier par un numéro.
}

\end{document}

