\documentclass{CR-projet}

%Import des packages utilisés pour le document
\usepackage[T1]{fontenc}


\definecolor{gris}{rgb}{0.95, 0.95, 0.95}
\definecolor{vert}{rgb}{0.14, 0.69, 0.11}

%Variables
\logo{logo_univ.png}
\title{Compte rendu - Réunion Client}
\author{Matthieu \bsc{FIN}}
\projet{Projet PGP}
\projdesc{Étude et implantation d'un outil graphique de gestion de clé PGP}
\filiere{Master 1 SSI }
\matiere{Conduite de projet}
\date{17 mars 2015}

\context{Réunion hebdomadaire}

\presentry{\bsc{Balmelle}}{Pierre}{\textbf{Présent}}{vert}
\presentry{\bsc{Barbay}}{Lucas}{\textbf{Absent}}{red}
\presentry{\bsc{Barry}}{Ibrahima Sory}{\textbf{Absent}}{red}
\presentry{\bsc{Bouillie}}{Bertille}{\textbf{Absente}}{red}
\presentry{\bsc{Fin}}{Matthieu}{\textbf{Présent}}{vert}
\presentry{\bsc{Leroy}}{Guillaume}{\textbf{Absent}}{red}
\presentry{\bsc{Thibault}}{Olivier}{\textbf{Présent}}{vert}

% -- Début du document -- %
\begin{document}

%Page de garde
\maketitle
\newpage
%La table des matières
%\tableofcontents

\newpage

\section{Redéfinitions des commandes obligatoires}

Liste des commandes à implanter :
\begin{itemize}
	\item Suppression de clés.
	\item Suppression de signature de base. 
	\item Suppression d'un uid. 
	\item Suppression d'une sous clef. 
	\item Révocation d'une clé. 
	\item Ajout d'une sous clef.
	\item Ajout d'un uid.
	\item Signature de clefs (basique).
	\item Clef par defaut.
	\item Afficher/modifier la preference d'algo d'un uid.
	\item Changer la date d'expiration.
	\item Changer la confiance.
	\item passwd (changer la passphrase)
	\item Pouvoir déchiffrer un fichier.
	\item Vérifier les signatures.
	\item Chiffrer un fichier.
	\item Signature de fichier.
\end{itemize}
\vspace{1cm}
Liste des commandes optionnelles :
\begin{itemize}
	\item Révocation de signature.
	\item disable/enable
	\item Définition d'un revokeur.
	\item clean (supprimer toutes les signature d'une clé).
	\item minimize (similaire a clean).
	\item [cross-certify]
	\item Révocation d'un uid.
	\item Révocation d'une sous clef.
\end{itemize}

\section{Actions}

\begin{itemize}
	\item La toile de confiance doit être complètement fonctionnel,
	sans représentation graphique pour le moment.
	\item Tout doit être libre de droits (les documents et images notamment) dans le cas contraire il faut 
	absolument citer les sources.
	\item Les absents doivent envoyer un mail sur la liste de diffusion avec leur progression sur leurs tâches.
	\item Revu des documents.
	\item Réalisation d'un plan d'action pour redéfinir les tâches et l'attribution de ressources.
	\item Préparer un livrable digne de ce nom.
\end{itemize}

\end{document}

