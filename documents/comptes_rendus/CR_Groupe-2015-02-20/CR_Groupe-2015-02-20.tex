\documentclass{../../res/CR-projet}

%Import des packages utilisés pour le document
\usepackage[T1]{fontenc}
\usepackage{../../res/tikz-uml}

\definecolor{gris}{rgb}{0.95, 0.95, 0.95}
\definecolor{vert}{rgb}{0.14, 0.69, 0.11}

%Variables
\logo{../../res/logo_univ.png}
\title{Compte rendu - Réunion Groupe}
\author{Matthieu \bsc{FIN}}
\projet{Projet PGP}
\projdesc{Etude et implantation d'un outil graphique de gestion de clé PGP}
\filiere{Master 1 SSI }
\matiere{Conduite de projet}
\date{20 février 2015}

\context{Revue de début de sprint 2}

\presentry{\bsc{Balmelle}}{Pierre}{\textbf{Présent}}{vert}
\presentry{\bsc{Barbay}}{Lucas}{\textbf{Absent}}{red}
\presentry{\bsc{Barry}}{Ibrahima Sory}{\textbf{Absent}}{red}
\presentry{\bsc{Bouillie}}{Bertille}{\textbf{Absente}}{red}
\presentry{\bsc{Fin}}{Matthieu}{\textbf{Présent}}{vert}
\presentry{\bsc{Leroy}}{Guillaume}{\textbf{Présent}}{vert}
\presentry{\bsc{Thibault}}{Olivier}{\textbf{Présent}}{vert}

% -- Début du document -- %
\begin{document}

%Page de garde
\maketitle
\newpage
%La table des matières
%\tableofcontents

\newpage

\section{Tâches définies}

Liste des tâches principales :
\begin{itemize}
  \item Ajout d'exceptions (Pour la communication GPG et pour assurer les préconditions définies pour chaque méthode).
  \item Communication GPG.
  \begin{itemize}
    \item Ajouter le mécanisme de communication interactive.
    \item Ajouter une méthode execute bloquante pour les interaction et une deuxième non bloquante.
  \end{itemize}
  \item Édition des clés.
  \item Gestion des clés.
  \item Mécanisme multi-instance.
  \item Toile de confiance.
  \item Configuration.
  \\
  \item Signature de messages.
  \item Vérification de signature.
  \item Chiffrement.
  \item Déchiffrement.
  \item Éditeur.
  \item Attaque.
  \item Étude GPG.
  \item Étude des limites open PGP / GPG.
\end{itemize}


\section{Diagramme des exceptions définies}

\begin{center}
  \begin{tikzpicture}
    \begin{umlpackage}{/}
      \umlclass[x=4]{QException}{}{} % QException
      \begin{umlpackage}[x=0,y=-2]{/Exception}
        \umlclass[x=0,y=-2]{IllegalArgumentException}{}{} % IllegalArgumentException
        \umlclass[x=8,y=-2]{IllegalStateException}{}{} % IllegalStateException
        \umlclass[x=4,y=0]{GPGException}{}{} % GPGException
        \umlclass[x=1,y=-4]{BadInteraction}{}{} % BadInteraction
        \umlclass[x=7,y=-4]{NotEnoughtInteraction}{}{} % NotEnoughtInteraction
      \end{umlpackage} % /
    \end{umlpackage} % /

    \umlinherit[geometry=--]{IllegalStateException}{QException}
    \umlinherit[geometry=--]{IllegalArgumentException}{QException}
    \umlinherit[geometry=--]{GPGException}{QException}
    \umlinherit[geometry=--]{BadInteraction}{GPGException}
    \umlinherit[geometry=--]{NotEnoughtInteraction}{GPGException}
  \end{tikzpicture}
\end{center}

\section{Attribution des tâches}

\subsection{Matthieu}

\begin{itemize}
  \item Communication GPG
  \item Ajout des exceptions.
  \item ajout d'un gestionnaire de signaux UNIX pour libérer les ressources en cas de fermeture non appropriée de l'application.
  \item Finalisation de la gestion de l'option -P dans le Launcher.
\end{itemize}

\subsection{Olivier}

\begin{itemize}
  \item Définition du chargement du premier profil par le Launcher.
  \item Modification de la gestion du profil par défaut dans le fichier de configuration
  à l'aide d'un nœud dédié plutôt qu'une répétition par profils.
  \item Affichage du nom du profil dans la barre d'état de la fenêtre principale.
  \item Modification de DialogProfile avec un constructeur prenant une Mainwindow
  et un prenant une Configuration.
\end{itemize}

\subsection{Pierre}

\begin{itemize}
  \item Gestion des clés.
  \item Importation de clé depuis un fichier ou un serveur de clés.
  \item Exporter une clé depuis un fichier ou un serveur de clés,
  qui doit se faire a partir de l'arbre de clés de la fenêtre principale
  (avec possibilité de sélection multiple).
  \item Modéliser la notion de serveur de clé.
  \item Possibilité de définir l'url du serveur de clés a utiliser.
  \item La gestion des clés .
\end{itemize}

\subsection{Guillaume}

Finitions de l'arbre des clés de la fenêtre principale.
\begin{itemize}
  \item Affichage en couleur des clés en fonction de leur validité
  (barrer/griser la clé si elle est expirée/révoquée).
  \item Afficher les clés publique dont on détient la clé privée en gras.
  \item Gestion des timestamp nul pour éviter d'avoir une clé valide qui s'affiche en expiration le 01/01/1970.
  \item Ajout d'une colonne affichant la confiance en la clé.
  \item Afficher les signatures des clés.
\end{itemize}

\subsection{Ibrahima}

\begin{itemize}
  \item Rédaction de l'étude sur GPG.
\end{itemize}

\subsection{Lucas}

\begin{itemize}
  \item Attaque (en attente d'échanges pour mieux définir l'attribution).
\end{itemize}

\section{Ajout de spécification a la classe de configuration}


Ajout d'un signal sgProfilesChanged
permettant d'indiquer le changement d'au moins un profil.

\begin{itemize}
  \item 
setDefaultProfile(Profile* p) :\\
préconditions : p != nullptr\\
postconditions : émission d'un signal sgProfilesChanged()
  \item
delete(Profile* p) :\\
préconditions : getProfileById(p->getId()) == p\\
postconditions : p == nullptr \&\& getProfileById((old p)->getId()) == nullptr\\
                 émission d'un signal sgProfilesChanged()
  \item
addProfile(Profile* p) :\\
préconditions : getProfileById(p->getId()) == nullptr\\
postconditions : getProfileById((old p)->getId()) == p\\
                 émission d'un signal sgProfilesChanged()
\end{itemize}

\end{document}

