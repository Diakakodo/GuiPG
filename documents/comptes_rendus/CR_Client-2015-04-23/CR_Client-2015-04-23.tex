\documentclass{CR-projet}

%Import des packages utilisés pour le document
\usepackage[T1]{fontenc}


\definecolor{gris}{rgb}{0.95, 0.95, 0.95}
\definecolor{vert}{rgb}{0.14, 0.69, 0.11}

%Variables
\logo{logo_univ.png}
\title{Compte rendu - Réunion Client}
\author{Matthieu \bsc{FIN}}
\projet{Projet PGP}
\projdesc{Étude et implantation d'un outil graphique de gestion de clé PGP}
\filiere{Master 1 SSI }
\matiere{Conduite de projet}
\date{23 avril 2015}

\context{Livraison 2}

\presentry{\bsc{Balmelle}}{Pierre}{\textbf{Présent}}{vert}
\presentry{\bsc{Barbay}}{Lucas}{\textbf{Présent}}{vert}
\presentry{\bsc{Barry}}{Ibrahima Sory}{\textbf{Présent}}{vert}
\presentry{\bsc{Bouillie}}{Bertille}{\textbf{Absente}}{red}
\presentry{\bsc{Fin}}{Matthieu}{\textbf{Présent}}{vert}
\presentry{\bsc{Leroy}}{Guillaume}{\textbf{Absent}}{red}
\presentry{\bsc{Thibault}}{Olivier}{\textbf{Présent}}{vert}

% -- Début du document -- %
\begin{document}

%Page de garde
\maketitle
\newpage
%La table des matières
%\tableofcontents

\newpage

\section{Sujets}

\begin{enumerate}
	\item La spécification technique du besoin nécessite quelques modifications :
	\begin{itemize}
		\item Dans l'objet l'équipe est composée de sept personnes.
		\item Ajouter les nouvelles fonctionnalité dans la STB (chiffrement/déchiffrement et vérification de signature d'un fichier).
		\item Le Cas d'utilisation 1 : Reformuler la description (l'application fait interface entre l'utilisateur et gpg).
		\item Cas d'utilisation 2 : Il y a deux fois l'expression "dans son trousseau".
	\end{itemize}
	\item Retour sur l'application guipg Actions a terminer pour validation de l'application :
	\begin{itemize}
		\item Il faut pouvoir signer une clé quand il y a plusieurs uid attaché a cette clé (on signe alors tous les uids de la clé).
		\item Avant de signer une clé il faut afficher les informations de la clé que l'on va signer (fpr, uid, nom, commentaire, email). 
		\item Chiffrement de fichier : Il faut pouvoir sélectionner plusieurs destinataires (uid).
		\item Déchiffrement d'un fichier : Afficher le contenu du fichier déchiffré 
		dans l'application uniquement si ce fichier et un fichier texte.
		\item Mettre le fichier de configuration de l'application dans ~/.guipgconf.xml (déjà opérationnel)
	\end{itemize}
	\item Étude : Non validée en l'état.
	\item Attaque : Non validée en l'état.
\end{enumerate}

\section{Actions}

\begin{enumerate}
	\item Préparation d'un rapport final pour le 22/05/2015 (Olivier, Matthieu, Pierre, Lucas, Ibrahima).
	\begin{itemize}
		\item Ce rapport servira a préparer la soutenance.
		\item Il doit donc rappeler le projet initial, les différentes étapes/problèmes/solutions traversé.
		\item Présenter les différents aspects techniques.
		\item L'état du projet final.
		\item Retour d'expérience.
	\end{itemize}
	\item Les modifications de la STB doivent être faite pour le 24/04 (Olivier).
	\item Les modifications de l'application doivent être faite avant le 30/04 (Olivier, Matthieu).
	\item Modifier le CDR (Pierre, Lucas)
	\begin{itemize}
		\item Indiquer que le test sur les signature ne fonctionne pas.
		\item Réaliser de nouveaux tests pour s'assurer des dernières fonctionnalités.
		\item Prévoir des tests concernant l'attaque.
	\end{itemize}
	\item Étude (Ibrahima)
	\begin{itemize}
		\item La finaliser au plus vite pour l'envoyer sur la liste de diffusion (Ibrahima).
		\item Tenir compte des retours.
	\end{itemize}
	\item Attaque (Lucas)
	\begin{itemize}
		\item Fournir une clé et sa seconde pré-image forgée grâce au script.
		\item La seconde pré-image doit être valide et utilisable par gpg.
	\end{itemize}
\end{enumerate}

\end{document}

