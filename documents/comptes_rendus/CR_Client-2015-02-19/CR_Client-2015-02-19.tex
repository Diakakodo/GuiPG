\documentclass{CR-projet}

%Import des packages utilisés pour le document
\usepackage[T1]{fontenc}


\definecolor{gris}{rgb}{0.95, 0.95, 0.95}
\definecolor{vert}{rgb}{0.14, 0.69, 0.11}

%Variables
\logo{logo_univ.png}
\title{Compte rendu - Réunion Client}
\author{Matthieu \bsc{FIN}}
\projet{Projet PGP}
\projdesc{Etude et implantation d'un outil graphique de gestion de clé PGP}
\filiere{Master 1 SSI }
\matiere{Conduite de projet}
\date{\today}

\context{Livraison 1er livrable fin du sprint 1}

\presentry{\bsc{Balmelle}}{Pierre}{\textbf{Présent}}{vert}
\presentry{\bsc{Barbay}}{Lucas}{\textbf{Absent}}{red}
\presentry{\bsc{Barry}}{Ibrahima Sory}{\textbf{Absent}}{red}
\presentry{\bsc{Bouillie}}{Bertille}{\textbf{Absente}}{red}
\presentry{\bsc{Fin}}{Matthieu}{\textbf{Présent}}{vert}
\presentry{\bsc{Leroy}}{Guillaume}{\textbf{Absent}}{red}
\presentry{\bsc{Thibault}}{Olivier}{\textbf{Présent}}{vert}

% -- Début du document -- %
\begin{document}

%Page de garde
\maketitle
\newpage
%La table des matières
%\tableofcontents

\newpage

\textbf{\Large \textcolor{red} {Livraison non validée !}}


\section{Modifications sur le livrable à prendre en compte}

\begin{itemize}
	\item Afficher le nom du profil courant sur la fenêtre principale (dans la barre de titre).
	\item Afficher les clés en couleur en fonction de leur validité.
	\item Afficher les signatures des clés et la confiance accordée a chaque clés (séparé de la validité).
	\item Pour les versions gpg2 quand le timestamp est nul ne pas afficher la date plutôt que d'afficher le 1 janvier 1970.
	ce cas correspond aux clé sans dates de fin de validité.
\end{itemize}

\section{Actions}

\begin{itemize}
	\item Report des études a plus tard. Priorité au \textbf{développement}.
	\item Attribuer \textbf{une} personne sur la tâche de l'attaque.
	\item Finir la gestion fine des clés au plus vite\\
	en particulier la communication interactive avec GPG.\\
	Pour ainsi terminer toutes les opérations de gestion de clés.
	\item Repenser l'architecture pour séparer complètement les tâches de développement.\\
	L'architecture doit être repensée impérativement pour la prochaine réunion du mardi 24 février 2015.
	Premières tâches majeures a distribuer :
	\begin{itemize}
		\item Chiffrement/déchiffrement de fichiers, ...
		\item Toile de confiance.
		\item Gestion des clés + noyau de l'application.
	\end{itemize}
	\item Réaliser un guide GIT. Et séparer les grandes tâches par branches.
	\item \textbf{Chaque semaines} :
	\begin{itemize}
		\item \textbf{Bilan} des tâches effectuées de la semaines passée.
		\item \textbf{Ré-attribution} des tâches pour la semaine suivante.
	\end{itemize}
	\item \textcolor{red} {Présence \textbf{obligatoire} de \textbf{chaque} membre aux réunions du mardi.}\\
	Si présence impossible :
	\begin{itemize}
		\item Envoie d'un \textbf{mail perso} a Magali Bardet.
		\item Contenant \textbf{l'avancement} personnel sur les tâches attribuées
		et la \textbf{prévision} des tâches de la semaine suivante.  
	\end{itemize}
	\item \textbf{Compte-rendu de chaque réunions} au format texte sur la liste de diffusion
	le \textbf{soir même} de la réunion ou le lendemain au plus tard.
	\item Avoir commencer la refonte nécessaire de l'architecture et la
	\textbf{distribution des nouvelles tâches} pour \textbf{Mardi 24 février 2015}.
\end{itemize}

\end{document}

