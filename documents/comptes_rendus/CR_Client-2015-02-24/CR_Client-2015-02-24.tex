\documentclass{CR-projet}

%Import des packages utilisés pour le document
\usepackage[T1]{fontenc}


\definecolor{gris}{rgb}{0.95, 0.95, 0.95}
\definecolor{vert}{rgb}{0.14, 0.69, 0.11}

%Variables
\logo{logo_univ.png}
\title{Compte rendu - Réunion Client}
\author{Matthieu \bsc{FIN}}
\projet{Projet PGP}
\projdesc{Etude et implantation d'un outil graphique de gestion de clé PGP}
\filiere{Master 1 SSI}
\matiere{Conduite de projet}
\date{24 février 2015}

\context{Réunion début de sprint 2}

\presentry{\bsc{Balmelle}}{Pierre}{\textbf{Présent}}{vert}
\presentry{\bsc{Barbay}}{Lucas}{\textbf{Présent}}{vert}
\presentry{\bsc{Barry}}{Ibrahima Sory}{\textbf{Présent}}{vert}
\presentry{\bsc{Bouillie}}{Bertille}{\textbf{Absente}}{red}
\presentry{\bsc{Fin}}{Matthieu}{\textbf{Présent}}{vert}
\presentry{\bsc{Leroy}}{Guillaume}{\textbf{Présent}}{vert}
\presentry{\bsc{Thibault}}{Olivier}{\textbf{Présent}}{vert}

% -- Début du document -- %
\begin{document}

%Page de garde
\maketitle
\newpage
%La table des matières
%\tableofcontents

\newpage

\section{Actions}

\begin{itemize}
	\item Mettre un fichier texte a la racine du repo git expliquant le fonctionnement des différentes branches
	où trouver les comptes rendus, le PDD...
	\item Attribution de l'étude et de l'implantation de l'attaque pour Lucas.\\
	Deux type d'attaques :
	\begin{itemize}
		\item Générer une première clé de test puis trouver une collision.
		\item Générer plusieurs clés et trouver une pair de clé qui provoque une collision.
	\end{itemize}
	\item Penser a un document d'attribution de tâches et de leur avancement
	pour pouvoir communiquer en interne et en externe permettant de voir qui fait quoi quand et quelle tâche pose éventuellement problème.\\
	Du genre todo-list améliorée ou diagramme de GANTT.
	\item Définir un chemin critique.
	\item Chacun met a jour l'avancement de ses tâches sur le planning.
	\item Ajouter une tâche d'empaquetage pour la livraison.
	\item Rédiger des tests.
	\item Rédiger un cahier de recettes (listes de tests réalisé par rapport a ce qui a était prévu).
	\item Tenir a jour \textbf{tous} les documents.
	\item Livrer le programme réalisant tous les tests.
\end{itemize}

\end{document}

