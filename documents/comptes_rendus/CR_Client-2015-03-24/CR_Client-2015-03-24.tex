\documentclass{CR-projet}

%Import des packages utilisés pour le document
\usepackage[T1]{fontenc}


\definecolor{gris}{rgb}{0.95, 0.95, 0.95}
\definecolor{vert}{rgb}{0.14, 0.69, 0.11}

%Variables
\logo{logo_univ.png}
\title{Compte rendu - Réunion Client}
\author{Matthieu \bsc{FIN}}
\projet{Projet PGP}
\projdesc{Étude et implantation d'un outil graphique de gestion de clé PGP}
\filiere{Master 1 SSI }
\matiere{Conduite de projet}
\date{24 mars 2015}

\context{Réunion hebdomadaire}

\presentry{\bsc{Balmelle}}{Pierre}{\textbf{Présent}}{vert}
\presentry{\bsc{Barbay}}{Lucas}{\textbf{Présent}}{vert}
\presentry{\bsc{Barry}}{Ibrahima Sory}{\textbf{Présent}}{vert}
\presentry{\bsc{Bouillie}}{Bertille}{\textbf{Absente}}{red}
\presentry{\bsc{Fin}}{Matthieu}{\textbf{Présent}}{vert}
\presentry{\bsc{Leroy}}{Guillaume}{\textbf{Absent}}{red}
\presentry{\bsc{Thibault}}{Olivier}{\textbf{Présent}}{vert}

% -- Début du document -- %
\begin{document}

%Page de garde
\maketitle
\newpage
%La table des matières
%\tableofcontents

\newpage

\section{Ordre d'implantation des commandes gpg revue avec le client}

\begin{enumerate}
	\item Changer la confiance (update-trustdb).
	\item Signature de clefs (basique).
	\item Ajout d'une sous clef.
	\item Ajout d'un uid.
	\item Pouvoir déchiffrer un fichier.
	\item Chiffrer un fichier.
	\item Signature de fichier.
	\item Vérifier les signatures fichiers.
	\item Clef par défaut.
	\item Afficher/modifier la préférence d'algo d'un uid.
	\item passwd (changer la passphrase).
	\item Suppression de clés.
	\item Suppression de signature de base.
	\item Suppression d'un uid.
	\item Suppression d'une sous clef.
	\item Révocation d'une clé. 
	\item Changer la date d'expiration.
\end{enumerate}

\section{Actions}

\begin{itemize}
	\item On abandonne toute la partie serveur.
	\item Réalisation de tests complets.
	\item Ajout du cahier de recette dans le livrable, avec possibilité de reproduire l'intégralité des tests.
	\item Pour l'attaque utiliser la version 4 (pour avoir la prise en compte des dates des clés sur le haché de celles-ci).
	\item Commencer l'implantation de l'attaque.
	\item 
\end{itemize}

\end{document}

