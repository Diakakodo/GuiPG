\section{La Qualité}

	%						%
	%  PREMIERE PARTIE		%
	%						%

\subsection{Présentation du Cahier des Recettes}

% Première slide %
\begin{frame}
  \frametitle{\color{white} Présentation}
  Liste des objets à test :
  \begin{itemize}
  \item L'interface graphique des fonctions GnuPG
  \item La visualisation de la toile de confiance
  \item L'implémentation de l'attaque sur les secondes pré-images
  \end{itemize}
\end{frame}

% Deuxième slide %
\begin{frame}
  \frametitle{\color{white} Environnement}
  \begin{itemize}
  \item Configuration..
  \item Outils..
  \item Outils..
  \item Jeux de données..
  \item Contraintes..
  \end{itemize}
  
\end{frame}


	%						%
	%  SECONDE PARTIE		%
	%						%

\subsection{Les tests}
% Troisième slide %
\begin{frame}
  \frametitle{\color{white} Stratégies des tests}
  Descriptions de l'approche des phases de test.
  Campagne de test.
  Ordre d'exécution des tests.
  Critères d'arrêt des tests.
\end{frame}

% Quatrième slide %
\begin{frame}
  \frametitle{\color{white} Gestion des anomalies}
  Procédure anomalie :
  \begin{itemize}
  \item Création mémo
  \item Ajout d'une entrée..
  \item Diffusion de la note..
  \item Désigner personne pour corriger..
  \end{itemize}
\end{frame}


