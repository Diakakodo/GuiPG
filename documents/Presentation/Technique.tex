\section{La technique}
\subsection{Configuration minimale}
\begin{frame}
  \frametitle{\color{white} Système d'exploitation}
  \begin{itemize}
    \item GNU Linux avec environement graphique
    \item Windows
  \end{itemize}
\end{frame}
\begin{frame}
  \frametitle{\color{white} Matériel}
  \begin{itemize}
    \item Processeur 1GHz
    \item 512 Mo de RAM
    \item 3 Mo d'espace disque
  \end{itemize}
\end{frame}
\subsection{Outils de développement}
\begin{frame}
  \frametitle{\color{white} Langage}
  \begin{itemize}
    \item C++ 11
    \item Qt 5
  \end{itemize}
\end{frame}
\begin{frame}
  \frametitle{\color{white} Pourquoi le C++ 11 ?}
  \begin{itemize}
    \item Rapidité d'exécution
    \item Communication simplifiée avec GPG
    \item Lanage natif de Qt
  \end{itemize}
\end{frame}
\begin{frame}
  \frametitle{\color{white} Pourquoi Qt 5 ?}
  \begin{itemize}
    \item Cadriciel multi-platformes
    \item KDE est basé sur ce cadriciel
  \end{itemize}
\end{frame}
\subsection{Architecture}
\begin{frame}
  \frametitle{\color{white} Patron de conception}
  \begin{itemize}
    \item Modèle - Vue
    \item Utilisé par Qt pour ses composants d'interface graphique
  \end{itemize}
\end{frame}
