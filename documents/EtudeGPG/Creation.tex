
\section{Gestion de clefs}
\subsection{Création des clefs}



la commande de generation de clefs gpg  \texttt{ ggp --gen-key } est une commande interactive:
En fonction des versions de gpg elle nous permet de creer divers types de clefs.\\
Dans les versions recentes elle nous offre 4 choix:\\
   (1) RSA and RSA (default)\\
   (2) DSA and Elgamal\\
   (3) DSA (sign only)\\
   (4) RSA (sign only)\\
   
Le prémier choix nous crée une paire de clef privée/publique de type RSA primaire qui servira à signer des messages et une seconde 
paire de clef privée/publique sécondaire qui servira à son tour à chiffrer et déchiffrer les messages.\\
Le sécond choix consiste à créer une paire de clef privée/publique primaire de type DSA pour les signatures et une paire de clef 
privée/publique secondaire de type El-gamal qui pour les opérations de chiffrement et de déchiffrement.\\
Le troisième et le quatrième crée respectvement une seule paire de clef privée/publique primaire de type DSA (respectivement RSA) pour
uniquement les signatures donc si cette option est choisi par l'utilisateur il devra lui même par la suite crée des clefs additionnelles 
pour pouvoir chiffrer et déchiffrer.\\

Il est important de préciser que dans les anciennes de versions de gpg , il n'existait pas le premier choix car à l'époque RSA
etait bréveté.
Le prémier choix et le second se valent du point de vue securité même si chacun des deux présente ses avanatages:\\
RSA:\\ 
-Géneration rapide des clefs
-commun, étudié et largement considéré comme sécurisé. 
DSA:\\
-Largement compatible avec GPG d'à peu près ne importe quelle version. 
-signature courte et plus pratiques.

La séconde question porte sur la taille des clefs de chiffrement / déchiffrement ( et de signature si vous avez choisi de générer une 
clef de signature seulement ).
Dans le cas absolu plus une taille de clef est grande plus la clef est securisée mais cette grandeur de la taille n'est pas sans 
conséquence car plus la taille est grande plus le temps de chiffrement augmente. Autrement dis il est mieux conseillé de prendre les
tailles par défaut proposées par le système puisque on peut toujours les editer ces clefs ( plus tard ) pour ajouter des sous-clefs en 
fonction de nouveaux bésoins. 
Pour les clefs de types RSA la taille varie entre 1024 et 4096 bits et la valeur par défaut choisie par gpg est 2048.
Pour les clefs de types DSA la taille varie entre 1024 et 3072 bits et la valeur par défaut est aussi de 2048 bits.
Pour Elgmal la taille varie entre 768 et 2048 bits .
Apres avoir choisi la taille l'étape suivante consiste à choisir la date d'expiration de la clef .
Cette date d'expiration peut être exprimé en jours , semaines , mois , années et on peux egalement choisir l'option par defaut qui 
donne une durée de vie illimité à notre clefs ce qui est le mieux consillé sachant qu'on peut aussi changé ces  dates avec l'option 
expire de gpg.

Les prochaines question servent à collecter des information qui serviront d'identifiant à votre pair de clef pour vous faciliter la
gestion de votre trousseax de clefs. Ces informations sont entre autre un nom pour la clef une adresse e-mail associé puis un commentaire 
sur la clef.Toutes ces informations peuvent être modifiées par la suite apres la génération de la paire de clefs.
Gpg utilise ensuite ces informations et un générateur de nombres pseudo-aléatoires pour générer la paire de clefs.

Il ne reste plus qu'à choisir mainténant que la phrase de passe qui va nous servir à chiffrer notre clef privée a fin de la stockée 
dans un endroit securisé . Cette phrase de passe sera demandé systématiquement à chauqe fois qu'on fera apel à notre clef privée pour
signer ou dechiffré.
En resumé la securité du systeme de chiffrement répose sur la clef secrète et la securité de cette dernière repose sur votre 
phrase de passe.
En  conséquence de cela il est primordiale de choisir une phrse de passe qui offre un minimum de securité dont entre autre:\\
-Elle dois etre la plus longue possible;
-Elle dois contenir des chiffres et des lettre et une lettre majuscule au debut
-Eviter des prénoms et noms ainsi que des date de naissaces. 

 
\subsection {Révocation d'une clef}
 Après la création de la paire de clefs, il faut créer un certificat de révocation pour la clef principale pour qu'au cas où la clef
venait à être compromise , que ce certificat soit publié pour dire à vos correspondants de ne plus chiffrer des messages à votre
destination en utilisant cette  clef. Par contre, même après publication de ce certificat de revocation la clef publique peut toujours 
être utilisée par vos correspondants pour vérifier les messages précédemment signés avec votre clef publique, mais elle ne peut plus 
être utilisée pour signer des messages qui vous seront destinés et également l'émetteur du certificat de revocation peut toujours 
s'il  le souhaite déchiffrer les messages qui ont été chiffrés avec cette clef publique en utilisant sa clef secrète.
Pour créer le certificat de révocation il faut ouvrir le terminal et lancer la commande : 

où revoque.asc est le fichier de révocation crée puis stocké dans le repertoire courant et ma\_clef est l'identifiant de la clef pour
aquelle on crée le certificat de révocation.
Cette commande est interactive et donc on est appelé a donner les motifs de creation du certificat de revocation avant confirmation et 
création du fichier revoque.asc .
Il est bien conseillé de stocker ce certificat dans un endroit tres sûr car quiconque a acces a ce certificat peut le diffuser a votre 
et rendre la clef non valide pour chiffrement même s'il est preferable de perdre son certificat de revocation que sa clef privée.
Lorsque vous devez revoquer votre clef il faut lancer depuis un terminal la commande:
\texttt{gpg --import revoke.asc } \\


\subsection { Afficher la listte des clefs publiques et privées}
\begin {itemize}
\item clef publique\\
La liste des clefs publiques de notre trousseau est accéssible depuis la commande gpg --list-keys

\item clef privée\\
Pour ce qui est des clefs privées cette liste est visible via la commande gpg --list-secret-keys

\subsection { Publication de clef}


\subsection { Ajout d'une clef publique au trousseau de clefs }



\subsection { Supprimer une clef du



