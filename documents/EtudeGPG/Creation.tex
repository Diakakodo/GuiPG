
\section{Gestion de clefs}
\subsection{Création des clefs}

La commande de génération de clefs gpg  \texttt{ ggp --gen-key } est une commande interactive:
En fonction des versions de gpg elle nous permet de créer divers types de clefs.\\
Dans les versions récentes, elle nous offre 4 choix:\\
   (1) RSA and RSA (default)\\
   (2) DSA and Elgamal\\
   (3) DSA (sign only)\\
   (4) RSA (sign only)\\
   
Le premier choix nous crée une paire de clefs privée/publique de type RSA primaire qui servira à signer des messages et une seconde 
paire de clef privée/publique secondaire qui servira à son tour à chiffrer et déchiffrer les messages.\\
Le second choix consiste à créer une paire de clefs privée/publique primaire de type DSA pour les signatures et une paire de clefs 
privée/publique secondaire de type El-gamal qui pour les opérations de chiffrement et de déchiffrement.\\
Le troisième et le quatrième crée respectivement une seule paire de clefs privée/publique primaire de type DSA (respectivement RSA) pour
uniquement les signatures donc si cette option est choisie par l'utilisateur il devra lui même par la suite créer des clefs additionnelles 
pour pouvoir chiffrer et déchiffrer.\\

Il est important de préciser que dans les anciennes versions de gpg , le premier choix n’existait pas, car à l'époque RSA
était breveté.
Le premier choix et le second se valent d’un point de vue sécurité même si chacun des deux présente ses avantages:\\
RSA:\\ 
\begin{itemize}
\item Génération rapide des clefs
\item Commun, étudié et largement considéré comme sécurisé.
\end{itemize}
DSA:\\
\begin{itemize}
\item Largement compatible avec GPG d'à peu près n’importe quelle version. 
\item Signature courte et plus pratique.
\end{itemize}

\medbreak

La seconde question porte sur la taille des clefs de chiffrement / déchiffrement (et de signature si vous avez choisi de générer une 
clef de signature seulement).
Dans le cas absolu plus une taille de clef est grande plus la clef est sécurisée, mais cette grandeur de la taille n'est pas sans
conséquence, car plus la taille est grande plus le temps de chiffrement augmente.
Pour les clefs de types RSA, la taille varie entre 1024 et 4096 bits et la valeur par défaut choisie par gpg est 2048.
Pour les clefs de types DSA, la taille varie entre 1024 et 3072 bits et la valeur par défaut est aussi de 2048 bits.
Pour Elgmal la taille varie entre 768 et 2048 bits.
Après avoir choisi la taille, l'étape suivante consiste à choisir la date d'expiration de la clef.
Cette date d'expiration peut être exprimé en jours , semaines , mois , années et on peut également choisir l'option par défaut qui 
donne une durée de vie illimitée à notre clef ce qui est le mieux conseillé sachant qu'on peut aussi changé ces  dates avec l'option 
expire de gpg.

Les prochaines questions servent à collecter des informations qui serviront d'identifiant à votre pair de clef pour vous faciliter la
gestion de votre trousseau de clefs. Ces informations sont entre autres un nom pour la clef une adresse e-mail associée puis un commentaire 
sur la clef. Toutes ces informations peuvent être modifiées par la suite après la génération de la paire de clefs.
Gpg utilise ensuite ces informations et un générateur de nombres pseudo-aléatoires pour générer la paire de clefs.

Il ne reste plus qu'à choisir maintenant la phrase de passe qui pourra nous servir, par exemple, à lire un mail chiffrer par notre clef publique. Cette phrase de passe sera demandée systématiquement à chaque fois qu'on fera appel à notre clef privée pour
signer ou déchiffrée.
En résumé la sécurité du système de chiffrement repose sur la clef secrète et la sécurité de cette dernière repose sur votre 
phrase de passe.

\newpage

Il est donc primordial de choisir une phrase de passe qui offre un minimum de sécurité :
\begin{itemize}
\item Elle doit être la plus longue possible;
\item Elle doit contenir des chiffres et des lettres et une lettre majuscule au début
\item Il faut éviter des prénoms et noms ainsi que des dates de naissance. 
\end{itemize}
 
\subsection {Révocation d'une clef}
 Après la création de la paire de clefs, il faut créer un certificat de révocation pour la clef principale dans le cas où la clef
venait à être compromise, que ce certificat soit publié pour dire à vos correspondants de ne plus chiffrer des messages à votre
destination en utilisant cette  clef. Par contre, même après publication de ce certificat de révocation la clef publique peut toujours 
être utilisée par vos correspondants pour vérifier les messages précédemment signés avec votre clef publique, mais elle ne peut plus 
être utilisée pour signer des messages qui vous seront destinés et également l'émetteur du certificat de révocation peut toujours 
s'il  le souhaite déchiffrer les messages qui ont été chiffrés avec cette clef publique en utilisant sa clef secrète.
Pour créer le certificat de révocation, il faut ouvrir le terminal et lancer la commande : \texttt{gpg --gen-revoke infoclef}

Où revoque.asc est le fichier de révocation crée puis stocké dans le répertoire courant et m’a\_clef est l'identifiant de la clef pour
laquelle on crée le certificat de révocation.
Cette commande est interactive et implique donc de donner les motifs de création du certificat de révocation avant confirmation et 
création du fichier revoque.asc .
Il est bien conseillé de stocker ce certificat dans un endroit très sûr, car quiconque a accès à ce certificat peut le diffuser à votre inssu
et rendre la clef non valide pour les opérations de chiffrement même s'il est préférable de perdre son certificat de révocation que sa clef privée.
Lorsque vous devez révoquer votre clef, il faut lancer depuis un terminal la commande:
\texttt{gpg --import revoke.asc } \\


\subsection {Afficher la liste des clefs publiques et privées}
\begin{itemize}
\item clef publique\\
La liste des clefs publiques de notre trousseau est accessible depuis la commande gpg --list-keys

\item clef privée\\
Pour ce qui est des clefs privées cette liste est visible par le biais de la commande gpg --list-secret-keys
\end{itemize}

\subsection{Publication de clef}

\subsection{Ajout d'une clef publique au trousseau de clefs}

\subsection{Supprimer une clef du trousseau}



