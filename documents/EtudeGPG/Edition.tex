\section{Édition de clefs}

la commande gpg --edit-key id\_clef 
présente un menu qui nous permet d’effectuer des tâches relatives aux clés:\\
\begin{enumerate}

\item Signer une clef:\\
Il est très important de bien vérifier l'identité des personnes dont vous signez les clefs. Si l'identité de la personne n'est pas 
vérifiée correctement, et que vous signez la clef, cela signifie que vous avez confiance en cette personne, et par conséquent que 
toute personne qui a confiance en vous a alors confiance en cette personne. Autrement en signant n'importe quelle clef, vous détruisez
votre réseau de confiance.

Une fois que vous êtes sûr d'avoir la bonne clef publique, vous pouvez à présent procéder à sa signature avec la commande.

\texttt{Command> sign}\\ 

\item Visualiser les signatures d'une clef\\
On peut verifier l'ensemble des signatures sur une clef avec la commande check
\texttt{Command> check }
\item Vérifier l'empreinte d'une clef:\\
Après avoir importé une clef publique, il est nécessaire, voire indispensable, de vérifier l'empreinte de la clef avec le fingerprint 
pour s'assurer que nous avons la bonne clef. Cette procédure se fait avec le propriétaire de la clef publique pour qu'ensemble vous 
comparez les résultats de vos empreintes si y a conformité. l'empreinte est vérifiée avec la commande:

\texttt{gpg> fpr\\
pub   2048R/81079E92 2015-02-05 mamou (mamouclef) <bb@y.com>\\
 Primary key fingerprint: A133 004C E06A 3DEC C8C6  26F5 593F 0033 8107 9E92} \\





\item Créer une signature locale\\
Avec la commande \texttt{lsign} on crée des signatures non exportable donc qu'on ne peut utiliser que localement.

\item Rendre une signature non révocable\\
Avec la commande \texttt{nrsign} la signature est marquée sur la clef comme non-revocable donc cette signature ne peut etre 
revoquée apres.

\item Signature locale et non révocable\\
Lorsque nous voulons créer une signature non revocable et non exportable ( utilisable en local uniquement ) on utilise \texttt{nrlsign}
 qui combien les deux precedentes.
\item Supprimer une signature\\
Il arrive qu'on signe la mauvaise clef et dans ce cas de figure nous pouvons la supprimer ( cette signature )  pour eviter de compromettre
 notre trousseaux de clefs, on utilise dans ce cas la commande \texttt{delsign} qui permet de supprimer une signature sachant bien
 entendu qu'une fois la signature deposée en public elle ne peut plus être  retractée.
\item Révoquer une signature\\
Lorsque nous voulons révoquer une signature on utilise la commande \texttt{revsig}.
\item Modifier les informations d'une clef\\
On peut bien evidemment ajouter un identifiant c'est à dire un nom une adresse email et un commentaire à notre clef avec 
la commande \texttt{adduid} comme on peut aussi  ajouter une photo à la clef avec la commande \texttt{addphoto} et également afficher 
cette photo à l'aide la commande \texttt{showphoto} puis enfin supprimer l'identifiant avec la commande \texttt{deluid}

-\item Afficher les préférences d'une clef\\
Avec la commande \texttt{pref} on peut lister les préférences de l'utilisateur selectionné sachant que la commande \texttt{showpref}
 donne beaucoup plus de details sur les préférences que la prémière.
\item Ajout de sous-clefs\\
les sous-clefs peuvent être considérées comme des paires de clefs séparées,mais automatiquement associées à votre paire de clefs 
principale donc essentiellement dediées au chiffrement et dechiffrement.
On peut ajouter une sous-clef à l'aide de la commande \texttt{addkey} comme on peut en supprimer avec la commande \texttt{delkey}.
Il est impossible de supprimer cette sous-clef avec \texttt{delkey} une fois qu'elle a été deposée sur un serveur de clefs.

\item Changer la date d'éxpiration d'une clef\\
Lorsque vous créer une clef gpg vous choisissez une date d'éxpiration et une fois l'échéance arrive alors que vous avez toujours bésoin 
d'utiliser cette clef alors dans ce cas vous pouvez modifier cette date pour rendre la clef encore utilisable pour un délai supplementaire
en fonction de vos bésoins.Cette modification  de la date d'expiration des sous-clefs se fait avec la commande  \texttt{expire}.

\item Degré de confiance d'une clef 
On peut attribuer un degré de confiance à une clef avec la commande \texttt{trust}.\\
  1 = je ne sais pas\\
  2 = ne fais pas confiance\\
  3 = je fais un peu confiance\\
  4 = je fais confiance\\
  5 = j'ai une confiance absolue ( choix le plus souvent reservé pour ses propres clefs)\\
  m = retourner au menu precedent\\
  

\item Activer et désactiver une clef
On peut aussi désactiver ou activer une clef respectivement avec la commande \texttt{disable} et \texttt{enable}.

\item Changer la phrase de passe\\
On peut aussi changer la phrase de passe de la clef secrète avec la commande \texttt{passwd}.


Cette liste est loin d'être complète donc pour plus d'information se reférer au man de gpg dans la section --edit-key.\\
\end{enumerate}
\section{Chiffrer et déchiffrer des documents}
  
Ggp fonctionne avec le principe de clef publique clef privée. Ce principe répose sur le fait que votre interlocuteur
utilise votre clef publique puis chiffre le message et vous envoi le chiffré que vous dechiffré en utilisant 
la clef sécrète correspondante.


\subsection {Chiffrement}:\\
Cette méthode permet de transmettre des données de façon illisible par le commun des mortels sauf la personne detentrice de la clef 
secrète pouvant dechiffrer le message pour enfin avoir accès au message clair.\\

la commande pour chiffrer un message est la suivante:\\
\texttt{gpg --armor --output "mon\_fichier\_chiffré" --encrypt "mon\_fichier\_à\_chiffrer"}
où on utilise l'option \texttt{--armor } pour pouvoir generer le fichier crypté en code ASCII à fin de pouvoir le transmettre en 
format texte et non obligatoirement en fichier joint uniquement.
Cette commande démande l'identifiant de la clef publique de la personne pour laquelle le message est adressé.


\subsection {Dechiffrement}:\\
A l'aide de la commande \texttt{gpg --output <mon\_fichier> --decrypt <mon\_fichier.gpg>} , on peut bien decryper le message 
mon\_fichier.gpg dont le contenu est mis en clair dans le fichier  mon\_fichier.\\

Cette commande commande demande la phrase de passe pour dechiffrer le message .


\section {Signer et vérifier des signatures} :\\

Une signature numérique d'un document nous permet de certifier et de dater un document. Si le document fait l'objet d'une quelconque
modification une vérification de signature echouera donc on peut bien dire qu'une signature numérique contrairement à la signature 
manuscrite ne peut être contrefaite.

Cette signature utilise la paire de clef privée/publique comme le fait l'opération de chiffrement et de déchiffrement.
La personne qui signe le document le signe avec sa clef privée pour vous l'envoyer ensuite vous utilisez sa clef publique
pour certifier exactement que le document a bien été signé par lui donc son conténu n'a pas été alteré par une quelconque 
autre personne.Une conséquence de l'utilisation des signatures numériques est qu'il est difficile d'infirmer que vous avez fait
une signature numérique car cela impliquerait que votre clé privée a été compromise.
Pour générer une signature numerique la ligne de commande est la suivante :

 \texttt{gpg --output doc\_signe --sign doc\_a\_signer } où doc\_signe est le document signé et doc\_a\_signer est le document 
 dont on souhaite signé.\\
 
 Cette commande nous démande la phrase de passe pour enfin signé le document donc on ne peut plus nié être à l'origine du 
 document.
 
 Un fois le document signé on peut soit avoir  bésoin juste de verifier la signature  ou d'aller un peu plus loin en vérifiant
 la signature du document mais aussi à extraire son conténu.
 Pour simplement vérifier la signature on utilise l'option \texttt{--verify} et pour vérifier et extraire en même temps on utilise l'option
  \texttt{--decrypt }.
 Dans cet exemple je vais montré comment verifier la signature et extraire en même temps le document avec la commande suivante:
  \texttt{gpg --output doc --decrypt doc.sig}  où doc est le fichier decrypé mis dans le repertore courant et doc.sig est le document signé.
 

 






