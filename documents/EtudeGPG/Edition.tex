\section{Édition de clefs}

 \texttt{--edit-key}\\
              Présente un menu qui nous permet d'effectuer toutes sortes de tâches relatives aux clés :



              \texttt{uid n }\\ 
               sélectionner le nom d'utilisateur N.
             Utilisez * pour sélectionner tous et 0 pour désélectionner tous.


              \texttt{key n  }\\ 
               sélectionner la sous-clé N. Utilisez * pour
              tout sélectionner et 0 pour désélectionner tous.
              
              \texttt{ check }\\ 
              
              vérifier les signatures


               \texttt{sign }\\   
              Apposer une signature sur les noms d'utilisateurs sélectionnés. Si la clef n'est pas déjà signée
              par l'utilisateur par défaut (ou par les utilisateurs spécifiés avec -u), le programme ré-affiche 
              les informations sur la clé, ainsi que son empreinte, et demande si elle doit être signée. 
              Cette question est répétée pour tous les utilisateurs spécifiés avec -u.


               \texttt{lsign }\\  
              Il fonctionne presque de la meme façon que --sign, mais la signature est marquée comme étant non-exportable et ne sera par conséquent 
            jamais utilisée par d'autres. Cela peut être utilisé pour créer des clés valides uniquement dans l'environnement local.
              
              \texttt{nrsign }\\ 
             Même principe également que --sign, mais la signature est marquée comme étant non-révocable et 
             ne peut dès lors jamais être révoquée.
            
              \texttt{nrlsign}\\ 
            Combine les fonctionnalités de nrsign et de lsign pour créer une signature à la fois non-révocable et non-exportable.


               \texttt{tsign }\\ 
              Faire une signature de confiance. Ce est une signature qui combine les notions de certification 
              et la confiance.Cette commande  est généralement utile dans les communautés ou groupes distincts.

              \texttt{delsig }\\ 
             Permet de supprimer une signature. Sachant bien qu'il n'est pas possible de
                      rétracter une signature une fois qu'elle a été déposée au public
                      (c'est a dire sur un serveur de clés).

              
               \texttt{revsig }\\ 
              Permet de révoquer une signature. Pour chaque signature qui a été
                      générée par l'une des clés secrètes, GnuPG demande si
                      un certificat de révocation doit être généré.


               \texttt{check  }\\ 
              Vérifie les signatures sur tous les ID d'utilisateurs sélectionnés.


               \texttt{adduid }\\ 
              Permet de créer un ID utilisateur supplémentaire.


               \texttt{addphoto}\\ 
                  Permet d'ajouter une photo d'identité . Cette commande demandera un
                      fichier JPEG qui sera intégré dans l'identifiant de l'utilisateur. 
                      Remarque :Un très grand fichier JPEG fera une très grande clé.
                      Notez également que certains programmes affichent votre  fichier JPEG
                      inchangé (GnuPG), et certains programmes l'affiche dans une boite de dialogue(PGP).


               \texttt{showphoto}\\ 
                     
                     Affiche la photo d'identité de l'utilisateur selectionné.


               \texttt{deluid }\\ 
             Supprimer un identifiant utilisateur ou la photo de l'utilisateur. Notez bien qu'il
                     n'est pas possible de retirer un identifiant utilisateur, une fois qu'il a été
                     rendu public (c'est à dire deposé sur un serveur de clefs). Dans ce cas il
                     es mieux conseillé d'utiliser revuid.


               \texttt{revuid }\\ 
    
              Revocation d'un  identifiant utilisateur ou de sa photo .

               \texttt{primary}\\ 
                      marquer le nom d'utilisateur sélectionné comme principal


               \texttt{keyserver}\\ 
                     indiquer l'URL du serveur de clés préféré pour les identifiants utilisateur sélectionnés
               \texttt{notation}\\ 
                     indiquer une notation pour les identifiants utilisateur sélectionnés


               \texttt{pref   }\\ 
                    lister les préférences de l'utilisateur selectionné 

               \texttt{showpref}\\ 
                     lister les préférences de l'utilisateur selectionné  (beaucoup plus detaillé que la prémière)
                     
                    

               \texttt{setpref }\\ 
                     
                indiquer la liste des préférences pour le nom d'utilisateur
                sélectionné


               \texttt{addkey }\\ 
             
             Permet d'ajouter une sous-clef


               \texttt{addcardkey}\\ 
                    ajouter une clef à une carte à puce


               \texttt{keytocard}\\ 
                      déplacer une clé vers une carte à puce

               \texttt{bkuptocard} "fichier" \\
                     déplacer une clé de sauvegarde vers une carte à puce


              \texttt{ delkey}\\ 
              Supprimer les sous-clefs sélectionnées tout en sachant que cette operation n'est pas possible une fois la sous-clef est 
              deposée sur un serveur de clef
              
              

               \texttt{revkey }\\ 
               révoquer la clef ou les sous-clefs sélectionnées


               \texttt{expire }\\ 
              changer la date d'expiration de la clé ou des sous-clés sélectionnées

               \texttt{trust  }\\ 
              Permet de changer la confiance pour une clef
              
            

               \texttt{disable }\\ 
              Desactiver la clef

               \texttt{enable }\\ 
              Activer la clef
              
               \texttt{addrevoker}\\ 
                     ajouter une clé de révocation

               \texttt{passwd }\\ 
              changer la phrase de passe de la clef secrete.


               \texttt{toggle }\\ 
               passer de la liste des clés secrètes à celle des clés privées
               et inversement


               \texttt{clean }\\ 
               Compacter les identifiants utilisateur inutilisables et retirer les signatures inutiles de la clef

               \texttt{minimize}\\ 
                     Compacter les identifiants utilisateurs inutilisables et retirer toutes les signatures de la clef.


            

               \texttt{save   }\\ 
        
               Sauvegarder tous les changements dans les trousseaux de clefs et quitter le programme.

               \texttt{quit   }\\ 
              quitter sans enregistrer les changements dans les trousseaux de clefs\\
              
Le listage vous montre la clé accompagnée de ses clés secondaires et de tous ses ID d'utilisateur. 
Les clés ou ID d'utilisateurs sélectionnés sont indiqués par un astérisque. 
Deux valeurs de confiance accompagnent la clé primaire : la première est la confiance dans le propriétaire, 
et la seconde est la valeur de confiance calculée. Des lettres sont utilisées pour les valeurs suivantes :\\

         \texttt{-}\\ 
            Aucune confiance dans le propriétaire attribuée / pas encore calculée.\\
         \texttt{e}\\ 
            Échec du calcul du niveau de confiance ; probablement dû à une clé expirée.\\
         \texttt{q}\\ 
            Pas assez d'informations pour le calcul.\\
         \texttt{n}\\ 
            Ne jamais faire confiance à cette clé.\\
         \texttt{m}\\ 
            Confiance marginale.\\
         \texttt{f}\\ 
            Confiance complète.\\
         \texttt{u}\\ 
            Confiance absolue.\\

