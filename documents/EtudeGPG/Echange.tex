\section{Échange de clefs}

Exporter une clef :
    Exporter une clef permet de la stocker dans un endroit pour la réutiliser sur un autre ordinateur.
    Soit on exporte toutes les clefs de tous les trousseaux ou si il y a au moins un nom donné, celles du nom donné. Les clefs exportées sont écrites
    sur STDOUT ou dans le fichier donné avec l'option --option ou -o.
    La commande gpg est \texttt{"gpg -\-export"}.

Envoyer une clef sur un serveur de clefs :
    Envoyer une clef permet de la rendre publique et facilement retrouvable pour ceux qui la voudraient.
    Cette fonction est équivalente à l'exportation de clefs mais permet en plus de l'envoyer sur un serveur de clefs. On peut préciser le
    serveur de clefs voulu. Il ne faut pas envoyer le trousseau entier, mais plutôt uniquement les nouvelles clefs ou celles qui ont été modifiées
    depuis le dernier envoi.
    La commande gpg est --send-keys key IDs.

Exporter clef secrète :
    Comme l'exportation de clefs mais avec les clefs secrètes à la place. Noter que l'exportation d'une clef secrète peut être un risque au niveau
    sécurité si la clef exportée est envoyer par un canal non sécurisé.
    Commande gpg : --export-secret-keys


Exporter les sous-clefs secrètes :
    Il est possible d'exporter uniquement les sous-clefs d'une clef, en neutralisant la partie secrète de la clef primaire. Cependant cette fonctionnalité est une extension GNU d'openPGP et les autres implémentations peuvent ne pas être compatibles avec une telle clef.
    Commande gpg : --export-secret-subkeys 

Importer une clef :
    Importe ou fusionne une ou plusieurs clefs. Cette action ajoute les clefs données au trousseau.
    Il existe une option permettant de faire seulement la fusion des nouvelles signatures.
    Commande gpg : --import

Importer une clef d'un serveur de clefs : 
    Importe une clef donnée d'un serveur de clefs. Si le nom du serveur n'est pas mentionné, celui par défaut est utilisé.
    Commande gpg : --recv-keys key IDs

Actualiser le trousseau de clefs : 
    Envoie une requête sur le serveur pour mettre à jour les clefs qui existe déjà dans le trousseau. Cette action est utile pour actualiser une clef
    avec les dernières signatures et ids d'utilisateur. Si aucun argument n'est donné, le trousseau entier est mis à jour. Si le trousseau contient une clef qui n'est pas reliée à un serveur, celui-ci doit être précisé.
    Commande gpg : --refresh-keys

Rechercher une clef sur un serveur de clefs : 
    Recherche le nom du serveur pour une clef donnée. Le serveur de clefs doit être spécifié. 
    Commande gpg : --search-keys names

Importer les clefs d'une URI donnée : 
    Importe les clefs stockées à l'URI spécifiée.
    Commande gpg : --fetch-keys URIs
