\section{Échange de clefs}

Pour pouvoir échanger avec nos correspondants, il faut bien leur faire parvenir notre clef publique pour qu'il puisse chiffrer des
messages à notre destination puis vérifier nos signatures en utilisant cette clef publique. À cet effet après la génération de la 
paire de clefs et la création du certificat de revocation correspondant, nous allons à présent exporter notre clef publique vers un 
serveur de clef publique (ou les transmettre à nos correspondants directement à travers un email ou sur un support amovible).

\subsection{Exporter une clef}
Avant de pouvoir envoyer nos clefs aux correspondants nous allons tout d'abord l'exporter dans notre répertoire avec l'option
 \texttt{--export } de gpg.\\
La commande est la suivante : \\


\texttt{gpg --output pb\_clef.gpg --armor --export id\_clef } où id\_clef est l'identifiant de la clef et pb\_clef.gpg est la clef 
publique. 

Une fois la clef publique exportée, vous pouvez la publier partout où vous le voulez que ce soit sur votre blog personnel ou sur les
serveurs de clefs publiques. Chacun peut alors accéder à votre clé publique et l'utiliser pour vous envoyer des messages puis vérifier 
vos signatures permettant l'authentification de vos messages.\\

On peut publier notre clef sur le serveur de clef à travers la commande:\\

\texttt{gpg --keyserver pgp.mit.edu --send-keys  mon\_id } dans cet exemple nous avons choisi le serveur de clef du du MIT
  (Massachusetts Institute of Technology)
\subsection{Importer une clé publique}


Pour pouvoir ajouter une clef publique à notre trousseau de clefs, il faut ajouter l'option --import  de gpg.

%\texttt{gpg --import pb_clef.gpg}

c) Supprimer ou retirer une clef publique de votre trousseau de clefs:
À l'aide de la commande \texttt{gpg --delete-key info-clef }, on peut supprimer une clef publique de notre trousseau de clefs.

À savoir également qu'avec l'option \texttt{--delet-secrete-key} on peut aussi supprimer la clef privée dont l'identifiant est en entrée.


