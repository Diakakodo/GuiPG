\section{Introduction}

GnuPG est un outil pour communiquer de manière sûre. Ce chapitre permet de couvrir l'ensemble des fonctionnalités 
importantes de GnuPG afin de pouvoir démarrer rapidement. Cela inclut la création, l'édition et l'échange de clefs.

GnuPG utilise la cryptographie à clef publique de façon à ce que les utilisateurs puissent communiquer de manière sûre. 
Dans un système à clef publique, chaque utilisateur possède une paire de clefs constituée d'une clef privée et d'une clef publique. 
La clef privée de l'utilisateur est gardée secrète, elle ne doit pas être révélée. La clef publique peut être distribuée à toute personne
avec qui l'utilisateur souhaite communiquer. GnuPG utilise un système un peu plus sophistiqué dans lequel un utilisateur possède 
une paire de clefs primaire et zéro ou plusieurs paires de clefs additionnelles. La clef primaire et les clefs additionnelles sont 
empaquetées pour faciliter la gestion des clefs. Un paquet de clefs peut être dans la plupart des cas considéré comme une simple paire 
de clefs.\\

GPG permet entre autres de:
\begin{itemize}
\item Chiffrer ses propres données.
\item Transmettre des données de toutes natures de façon authentique pour que seul le destinataire puisse l'ouvrir.
\item Signer des documents et donc s'identifier comme le véritable auteur d'un texte ou mail.
\item Stocker les clés de vos amis et créer une toile de confiance.
\end{itemize}