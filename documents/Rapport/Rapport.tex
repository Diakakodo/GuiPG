\documentclass{../res/univ-projet}

%Import des packages utilisés pour le document
\usepackage[utf8x]{inputenc}
\usepackage[francais]{babel}
\usepackage[T1]{fontenc}
%\usepackage{array}
%\usepackage{hyperref}
%\usepackage{tabularx, longtable}
%\usepackage[table]{xcolor}
%\usepackage{fancyhdr}
%\usepackage{lastpage}
\usepackage{../res/tikz-uml}
\usepackage{tikz}
\usepackage{calc}
\usepackage{xstring}
\usepackage{pgfopts}

\definecolor{gris}{rgb}{0.95, 0.95, 0.95}

%Redéfinition des marges
%\addtolength{\hoffset}{-2cm}
%\addtolength{\textwidth}{4cm}
\addtolength{\topmargin}{-1cm}
\addtolength{\textheight}{1cm}
\addtolength{\headsep}{0.8cm} 
\addtolength{\footskip}{-0.2cm}


%Import page de garde et structures pour la gestion de projet
%\usepackage{structures}

%Variables
\logo{../res/logo_univ.png}
\title{Rapport de projet}
\author{\\ Pierre \bsc{Balmelle},\\ Lucas \bsc{Barbay}, \\ Matthieu \bsc{Fin},
\\ Ibrahima \bsc{Sorry Barry},\\ Olivier \bsc{Thibault}}
\projet{Projet PGP}
\projdesc{Étude et implantation d'un outil graphique de gestion de clefs PGP}
\filiere{Master 1 SSI }
\matiere{Conduite de projet}
\date{\today}

% -- Début du document -- %
\begin{document}

%Page de garde
\maketitle
\newpage
\section{Remerciements}
Avant de débuter ce rapport, il nous parâit indispensable de remercier les 
personnes qui ont contribué au bon déroulement de ce projet.

Nous remercions tout d'abord notre cliente, madame Magali BARDET, pour sa 
participation, son attention et la confiance qu'elle nous a accordé au 
cours de notre mission.

D'autre part, nos remerciements vont aussi à monsieur Karim ABDELLAH 
GODARD ainsi que monsieur Rémi DIONISI, pour la formation, le suivit 
et le soutien qu'ils nous ont apporté au cours de l'année.

Enfin, nous remercions l'ensemble de l'équipe enseignante du département 
informatique de rouen, pour les connaisances qu'ils nous ont fournit lors 
de notre formation, sans quoi ce projet n'aurait pas eu lieu.
\newpage
%La table des matières
\tableofcontents

\newpage

\section{Introduction}

Au cours de notre première année de Master Informatique, un projet annuel 
en équipe nous a été confié afin de mettre en pratique ce qui nous a 
été enseigné dans l'unité ``Gestion de projet''. Il s'agit donc d'apprendre 
à s'organiser en équipe, faire preuve de méthodologie et d'adaptabilité 
pour mener à bien les objectifs d'un projet informatique. Ce rapport à pour 
but de présenter nos travaux durant ce projet.

Dans ce cadre, nous avons eu pour client Mme Magali BARDET, enseignante et 
responsable du Master Sécurité des Systèmes Informatiques (SSI) de l'UFR des 
Sciences et Techniques de Rouen. Notre mission s'est déroulé du 17 octobre 
2014 au 1\up{er} mai 2015. Durant cette période, nous avons été formés à la 
gestion de projet par M. Rémi DIONISI ainsi que M. Karim ABDELLAH GODARD.

Le thème de ce projet tourne autour d'OpenPGP. Le sujet se décompose en trois 
parties. La première partie consistait en la rédaction d'une étude d'OpenPGP 
et de son implantation en ligne de commande GnuPG. La deuxième partie a été 
l'implantation d'une interface graphique pour GnuPG. Enfin la troisième fut 
l'implantation et l'explication d'une attaque sur OpenPGP.


  \subsection{Contexte}
   % Sujet
  % Equipe
  % Cadre 
  Ce projet a été réalisé dans le cadre de la première année du Master Sécurité des Système d'informations enseigné à  l'Université de Rouen. Ce projet est découpé en deux parties et a commencé au début de l'année. Il nous a été demandé au premier semestre de rédiger les documents de Gestion de projets qui ont servi de base durant toute la durée du projet. Au second semestre il a fallu développer l'application en s'aidant des documents rédigés et en les faisant évoluer au fur et à mesure du projet.
  \subsection{L'équipe}
  L'équipe de ce projet était composée au début de 7 étudiants en première année du master évoqué ci-dessus : Pierre Balmelle, Lucas Barbay, Bertille Bouillie, Mathieu Fin, Guillaume Leroy, Ibrahima Sorry Barry et Olivier Thibault. Nous avons du réaliser ce projet dans le cadre du module de gestion de projet sous la tutelle de Monsieur Karim ABDELLAH GODARD, et grâce aux cours de Monsieur Remi DIONISI. Le sujet de ce projet nous a été donné par Magali Bardet, enseignante-chercheuse à l'Université de Rouen.

\section{Présentation du projet initial}

Pour introduire ce projet, nous allons commencer par parler du standard OpenPGP. Ce standard est un format de cryptographie normalisé dans la RFC 4880. OpenPGP décrit le format des messages qui sont adaptés aux outils permettent l'envoie sécurisés de message ou bien le stockage de message.
GnuPG (GNU Privacy Guard) est un de ces outils. Il se base sur le logiciel PGP et utilise un système hybride liant cryptographie symétrique et asymétrique pour permettre l'envoie de message chiffrés et/ou signés. Pour pouvoir s'échanger des messages, les utilisateurs de GPG doivent s'envoyer leur clé publique qui servira au chiffrement des messages.

  \subsection{Objectifs}
  L'objectif de ce projet est de développer un outil graphique de gestion de clefs PGP. Il existe déjà plusieurs éditeurs permettant d'utiliser GPG graphiquement mais aucun ne permettent une gestion fine des clefs et la partie toile de confiance n'est pas intuitive. Il nous est donc demandé de réaliser une interface qui soient plus complète que les outils existants. L'objectif est dans un premier temps d'étudier complètement GnuPG et OpenPGP pour comprendre parfaitement son utilisation et réaliser un logiciel le plus exhaustif possible. L'interface réalisée devra permettre aux utilisateurs de faire des réglages techniques qu'ils soient experts ou novices. Un document expliquant les fonctionnalités devra être livré avec le projet.
  Enfin il est demandé d'implanter l'attaque sur les KeyID décrite dans le magasine de sécurité informatique le MISC. Cette attaque est basée sur les mauvais usages de PGP par les utilisateurs.
  
  \newpage
  
  \subsection{Fonctionnalités}
  
  Les principales fonctionnalités du projet sont : \medbreak
  \begin{itemize}
  \item Exécution d'actions GPG \smallbreak
  Appel des actions via l'interface (création, modification, suppression..) \smallbreak
  \item Chiffrer / déchiffrer / signer / vérifier \smallbreak
  Chiffrer ou déchiffrer ou signer ou vérifier un message copié dans l'éditeur de l'interface. Il est possible d'exporter le résultat dans un fichier ou d'importer un fichier. Dans ce dernier cas, le résultat est affiché via l'interface. \smallbreak
  \item Affichage des commandes, des retours et des erreurs \smallbreak
  L'utilisateur peut choisir d'afficher ou non les commandes, les retours et les erreurs associés à chaque action GPG. \smallbreak 
  \item Choix du profil utilisateur \smallbreak
  L'utilisateur peut choisir son profil au lancement de l'interface via l'option -P ou en cours d'utilisation. Si au lancement l'option n'est pas lancée et qu'aucun profil par défaut n'est défini, l'utilisateur doit choisir un profil. Lors de l'installation, un profil par défaut est créé. \smallbreak
  \item Modification de la toile de confiance \smallbreak
  Modification de la toile de confiance par un changement de niveau de confiance, l'ajout d'une nouvelle clé ou la modification d'une clé. \smallbreak
  \item Calcul d'une seconde pré-image pour une clé donnée \smallbreak
  Attaque permettant d'obtenir une nouvelle clé contenant le même KeyId que la première.
  \end{itemize}
  
  L'attaque sur les KeyId est indépendante et ne fera pas partie de l'interface graphique. 
  
  \subsection{Contraintes}
  
  L'application doit fonctionner sur le système d'exploitation GNU/Linux, en particulier sous les environnements KDE et Gnome. La validité et la confiance doit être représentée sur l'interface par différents niveaux de couleurs.
  L'interface doit être à la fois pédagogique et précise pour faciliter l'utilisation de GPG.
  
  
  \subsection{Organisation}
  
  Pour l'organisation de ce projet, il nous a été demandé d'utiliser une méthode agile. Nous avons décidé de choisir scrum avec comme scrum master Olivier, comme chargés client Bertille et Ibrahima, comme architectes Mathieu et Guillaume et enfin Pierre et Lucas comme testeurs. Cette équipe est encadré par le propriétaire du produit Magali BARDET et le consultant en gestion de projet 	Karim ABDELLAH GODARD.
  
  \newpage

\section{Présentation du projet final}
  \subsection{Objectifs}
    Pendant le développement du projet, deux membres nous ont quittés (Bertille et Guillaume), ce qui nous a forcé à revoir toute notre planification, n'ayant plus les moyens humains pour tenir le rythme prévu.
  \subsection{Organisation}
    
    Le projet a été replanifié et découpé en deux livraisons.
    L'attribution des tâches a été refaite, avec la répartition suivante :
    \begin{itemize}
      \item Ibrahima sur l'étude de GPG,
      \item Lucas sur l'attaque sur les KeyID,
      \item Les autres (Matthieu, Olivier et Pierre) sur le développement de l'application.
    \end{itemize}

  \subsection{Fonctionnalités}
    Les fonctionnalités ayant été complétées sont :\medbreak
  \begin{itemize}
  \item Exécution d'actions GPG \smallbreak
  Création, exportation, importations de clés (en local)\smallbreak
  \item Chiffrer / déchiffrer / signer / vérifier \smallbreak
  Chiffrer ou déchiffrer ou signer ou vérifier un fichier. \smallbreak
  \item Affichage des commandes, des retours et des erreurs \smallbreak
  L'utilisateur peut choisir d'afficher ou non les commandes, les retours et les erreurs associés à chaque action GPG. \smallbreak 
  \item Création / Modification / Suppression du profil utilisateur \smallbreak
  \item Modification de la toile de confiance \smallbreak
  Modification de la toile de confiance par un changement de niveau de confiance, l'ajout d'une nouvelle clé ou la modification d'une clé. \smallbreak
  \item Calcul d'une seconde pré-image pour une clé donnée \smallbreak
  Attaque permettant d'obtenir une nouvelle clé contenant le même KeyId que la première.
  \end{itemize}


  

  Les fonctionnalités qui ont été supprimées par rapport au plan initial sont :\medbreak
  \begin{itemize}
  \item Toile de confiance graphique\smallbreak
  Représentation de la toile de confiance sous forme de graphe\smallbreak
  \item Chiffrer / déchiffrer / signer / vérifier \smallbreak
  Chiffrer ou déchiffrer ou signer ou vérifier le texte contenu dans l'éditeur de l'interface.\smallbreak
  \item Edition de clé \smallbreak
  Quelques fonctions d'édition de clé ne sont pas implémentées, comme la révocation d'une clé ou la signature avec une clé en particulier \smallbreak 
  

  La partie interface graphique a été validée par le client, mais malheureusement l'attaque et l'étude sur GPG n'ont pas été validés car ils n'ont pas été rendus à temps.

  \end{itemize}

\section{Les problèmes rencontrés}

  Lors de la réalisation de ce projet nous avons dû faire face a
  différents problèmes aussi bien d'ordre organisationnel que techniques,
  auxquels nous avons du remédier.

  \subsection{Problèmes organisationnels}

    Les difficultés organisationnelles rencontrées sont essentiellement
    survenues du fait de la modification de l'effectif de l'équipe durant la
    phase de réalisation du projet. 

    L'équipe initialement formée de sept développeurs,
    s'est trouvé réduite de deux personnes :
    \begin{itemize}
      \item Bertille Bouillie qui dès les deux premières semaines n'a plus donné
      aucune nouvelle, son abandon a seulement pu être officiellement pris en compte au bout d'un mois et demi.
      \item Guillaume Leroy qui a abandonné le projet au bout d'un mois et demi.
    \end{itemize}
    De plus les actions entreprises lors de l'avant projet n'ont pas suffi à l'ensemble de l'équipe 
    pour se former sur les différents outils utilisés lors de la phase de réalisation,
    puisque lors du premier sprint toutes les tâches attribuées a certains membres n'ont pas pu être
    réalisées suite à ces différentes lacunes.

    Nous avons, suite à cela pris du retard sur le premier sprint.

    Un plan d'action a alors été mis en place pour faire l'état des lieux des tâches
    faites, et des tâches réalisables avec le restant de l'équipe et le temps disponible.

    Ce plan a été présenté et discuté avec le client pour redéfinir le périmètre du projet.

    Suite à cela nous avons dû réattribuer les tâches à chacun pour pouvoir tenir la charge de travail,
    Lucas et Ibrahima ont souhaité se charger respectivement de l'attaque et de l'étude d'OpenPGP / GPG.

    Ainsi nous avons redéfini les tâches de développement de l'application
    entre Olivier, Matthieu (développeurs) et Pierre (Testeur).
    Les fonctionnalités de l'application ont également été redéfinies pour pouvoir être réalisables avec un effectif
    aussi réduit.

    De plus toute la réalisation de la visualisation de la toile de confiance a été réduite
    à l'affichage des niveaux de confiance d'une clé et non plus en dessin d'une toile de confiance graphique.

    Nous avons également à la fin du premier sprint suite à ces observations redéfinir les rôles de chaque membres : 

    \begin{tabular}{|l|l|}
      \hline
      \bfseries{Noms}      & \bfseries{Rôles}                             \\
      \hline
      Olivier Thibault     & Chef de projet / Développeur                 \\
      Matthieu Fin         & Architecte / Chargé client / Développeur     \\
      Pierre Balmelle      & Testeur                                      \\
      Lucas Barbay         & Chargé de l'attaque                          \\
      Ibrahima Sorry Barry & Responsable de documentation OpenPGP / GnuPG \\
      \hline

    \end{tabular}

    
  \subsection{Problèmes techniques}

    \subsubsection{Maîtrise des outils}

      Lors de la phase d'avant projet nous avons organisé des sessions d’entraînement afin
      de prévenir du risque lié a l’incompréhension des outils utilisés tels que git, Qt, le langage C++,
      ou même le logiciel GnuPG.

      Seulement même si lors de l'avant projet l'ensemble de l'équipe semblait maîtriser ces
      différents outils nous nous sommes rendu compte lors de la phase de développement que ce n'était pas le cas
      pour tout le monde.

      Les outils utilisés ont été définis conjointement dès le 27 novembre, or dès le début de la phase de développement
      (c'est à dire au mois de janvier) certain membres n'avaient toujours pas installé les dits outils. Par conséquent
      ils ne pouvaient les maîtriser. Nous avons donc dû nous réunir pour installer et former a minima les différents membres
      sur ces outils, ce qui nous a fait perdre du temps sur le premier sprint de développement.

    \subsubsection{Maîtrise de GnuPG}
    
    \subsubsection{Communication GnuPG}
  

\section{Bilan du projet}
  
  \subsection{Compétences acquises}

    \subsubsection{Technique}
    
      Ce projet nous a beaucoup appris sur la partie technique du développement d'une interface graphique, en particulier l'utilisation du cadriciel (framework) Qt sous sa version 5, et de l'environnement de développement associé Qt Creator.

      Nous avons également gagné des compétences en C++, qui a été utilisé pour développer notre interface graphique.

      Les tests ont été l'occasion d'acquérir des connaissances sur le module Qt Test, et sur la manière d'effectuer des tests en général.



    \subsubsection{Organisationnel}

      Nous avons retenu de ce projet qu'il est essentiel de s'assurer que l'équipe est effectivement formée sur les outils qui vont être utilisés ainsi que de faire l'inventaire de ces outils.
      Il faut aussi suivre régulièrement la progression de chaque membre dans la réalisation des tâches qui lui sont attribuées, pour être certain de l'état de progression de celles-ci et pouvoir s'adapter si elles ne sont pas finies à temps (ou finies plus tôt que prévu).




  \subsection{Axes d'amélioration}
  
    Le projet pourrait être amélioré en développant les commandes gpg prévues dans la STB initiale mais qui ont dûes être supprimées par manque de personnel, notamment pour l'édition de clé. 
    La toile de confiance graphique peut aussi être affichée sous forme de graphe, et offrir la possibilité de modifier une clé en effectuant un clic droit sur l'un des sommets du graphe (représentant une clé).
    La personnalisation de l'interface pourrait être développée, donnant par exemple l'opportunité de personnaliser les couleurs représentant la validité ou la confiance d'une clé, ou les informations affichées sur chaque clé.

  \subsection{Rétrospective}



\section{Conclusion}



\end{document}

