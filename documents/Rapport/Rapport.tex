\documentclass{../res/univ-projet}

%Import des packages utilisés pour le document
\usepackage[utf8x]{inputenc}
\usepackage[francais]{babel}
\usepackage[T1]{fontenc}
%\usepackage{array}
%\usepackage{hyperref}
%\usepackage{tabularx, longtable}
%\usepackage[table]{xcolor}
%\usepackage{fancyhdr}
%\usepackage{lastpage}
\usepackage{../res/tikz-uml}
\usepackage{tikz}
\usepackage{calc}
\usepackage{xstring}
\usepackage{pgfopts}

\definecolor{gris}{rgb}{0.95, 0.95, 0.95}

%Redéfinition des marges
%\addtolength{\hoffset}{-2cm}
%\addtolength{\textwidth}{4cm}
\addtolength{\topmargin}{-1cm}
\addtolength{\textheight}{1cm}
\addtolength{\headsep}{0.8cm} 
\addtolength{\footskip}{-0.2cm}


%Import page de garde et structures pour la gestion de projet
%\usepackage{structures}

%Variables
\logo{../res/logo_univ.png}
\title{Rapport de projet}
\author{\\ Pierre \bsc{Balmelle},\\ Lucas \bsc{Barbay}, \\ Matthieu \bsc{Fin},
\\ Ibrahima \bsc{Sory Barry},\\ Olivier \bsc{Thibault}}
\projet{Projet PGP}
\projdesc{Étude et implantation d'un outil graphique de gestion de clefs PGP}
\filiere{Master 1 SSI }
\matiere{Conduite de projet}
\date{\today}

% -- Début du document -- %
\begin{document}

%Page de garde
\maketitle
\newpage
\thispagestyle{empty}
\setcounter{page}{0}
\section*{Remerciements}
Avant de débuter ce rapport, il nous parâit indispensable de remercier les 
personnes qui ont contribué au bon déroulement de ce projet.

Nous remercions tout d'abord notre cliente, madame Magali BARDET, pour sa 
participation, son attention et la confiance qu'elle nous a accordée au 
cours de notre mission.

D'autre part, nos remerciements vont aussi à monsieur Karim ABDELLAH 
GODARD ainsi que monsieur Rémi DIONISI, pour la formation, le suivi 
et le soutien qu'ils nous ont apportés au cours de l'année.

Enfin, nous remercions l'ensemble de l'équipe enseignante du département 
informatique de Rouen, pour les connaissances qu'ils nous ont fournies lors 
de notre formation, sans quoi ce projet n'aurait pas eu lieu.
\newpage
\setcounter{page}{0}
%La table des matières
\tableofcontents

\newpage

\section{Introduction}

Au cours de notre première année de Master Informatique, un projet annuel 
en équipe nous a été confié afin de mettre en pratique ce qui nous a 
été enseigné dans l'unité ``Gestion de projet''. Il s'agit donc d'apprendre 
à s'organiser en équipe, faire preuve de méthodologie et d'adaptabilité 
pour mener à bien les objectifs d'un projet informatique. Ce rapport a pour 
but de présenter nos travaux durant ce projet.

Dans ce cadre, nous avons eu pour client Mme Magali BARDET, enseignante et 
responsable du Master Sécurité des Systèmes Informatiques (SSI) de l'UFR des 
Sciences et Techniques de Rouen. Notre mission s'est déroulée du 17 octobre 
2014 au 1\up{er} mai 2015. Durant cette période, nous avons été formés à la 
gestion de projet par M. Rémi DIONISI ainsi que M. Karim ABDELLAH GODARD.

Le thème de ce projet tourne autour d'OpenPGP. Le sujet se décompose en trois 
parties. La première partie consistait en la rédaction d'une étude d'OpenPGP 
et de son implantation en ligne de commande GnuPG. La deuxième partie a été 
l'implantation d'une interface graphique pour GnuPG. Enfin la troisième fut 
l'implantation et l'explication d'une attaque sur OpenPGP.

Dans un premier temps, il nous semble nécessaire de vous parler plus en 
détail du contexte de ce projet. Nous vous présenterons ensuite ce qui a été 
réalisé, puis nous ferons état des problèmes rencontrées. Enfin, pour finir 
nous effectuerons un bilan de ce projet.

\section{Contexte}
  \subsection{OpenPGP}
    OpenPGP est un un standard normalisé dans la RFC 4880 qui définit des 
    formats de messages. Ces formats reposent sur de la cryptographie hybride, 
    c'est à dire une combinaison de clefs symétriques et asymétriques. Ils 
    permettent de fournir des services de sécurité pour les communications 
    électroniques ainsi que le stockage de données. Ces services quand à eux,
    oeuvrent pour garantir:
    \begin{itemize}
     \item la confidentialité (seuls les destinataires du message 
	   peuvent le déchiffrer),
     \item l'authenticité (l'émetteur d'un message est bien celui qu'il 
	   prétend être),
     \item l'intégrité (l'état des données transmises est le même qu'au 
	   moment de l'envoi, elles n'ont pas été altérées).
    \end{itemize}
 
  \subsection{GnuPG}
    GnuPG est une implémentation libre du standard OpenPGP. Ce logiciel a 
    pour but de proposer une alternative au logiciel PGP qui lui, est une
    implémentation non libre. 
    GnuPG permet de signer, vérifier (assure l'authenticité et l'intégrité), 
    chiffrer et déchiffrer (assure la confidentialité) des fichiers, ainsi 
    que de gérer un trousseau de clefs cryptographiques (asymétriques) 
    permettant ces actions. 
    Un modèle de toile de confiance est associé au trousseau de clefs. 
    Ce modèle permet de gérer pour chaque clef du trousseau une 
    validité pour garantir qu'elle appartient bien à la personne 
    identifié, ainsi qu'un niveau de confiance en cette personne.
    
  \subsection{Le besoin}
    Il existe quelques interfaces graphiques (comme KGpg, GPA, Seahorse). 
    Cependant, ces interfaces ne permettent pas une gestion fine des clefs 
    et la partie toile de confiance n’est pas correctement représentée.
    Il nous a donc été demandé de proposer une interface graphique 
    permettant à un utilisateur novice en PGP, mais averti en sécurité, 
    de faire facilement des réglages techniques sur son trousseau et de pouvoir 
    mettre en place sa toile de confiance. Le logiciel doit donc être à la 
    fois pédagogique, précis et accompagné d'un document d’explications et 
    de recommandations d’utilisation. Il doit aussi fonctionner sous kde et 
    gnome.
  
    Nous devions aussi réaliser une étude complète d'OpenPGP et du 
    logiciel GnuPG (en ligne de commande), d’en comprendre le 
    fonctionnement détaillé et de rédiger un rapport illustrant toutes 
    les fonctionnalités.

    Enfin, nous devions effectuer des recherches sur les limites 
    cryptographiques de PGP et produire un document d’analyse de ces 
    limites. En particulier il était demander d'implanter l’attaque sur les 
    identifiants de clefs PGP décrite dans le numéro 75 de MISC magazine.
    
  \subsection{Acteurs du projet}
    La cliente de ce projet est Mme. Magali BARDET (enseignante et responsable 
    du master SSI à l'UFR des sciences et techniques de Rouen). 

    M. Karim Abdellah GODARD est un intervenant extérieur et a assuré le 
    rôle de formateur et consultant en gestion de projet.

    L'équipe en charge du développement était constituée de sept étudiants actuellement en première année du master IGIS de Rouen : 

    \begin{tabular}{ll}
    - Bertille BOUILLIE & Reponsable client \\
    - Guillaume LEROY & Architecte \\
    - Ibrahima Sory BARRY & Chargé client \\
    - Lucas BARBAY & Testeur \\
    - Matthieu FIN & Responsable technique \\
    - Pierre BALMELLE & Responsable qualité \\
    - Olivier THIBAULT & Chef de projet \\
    \end{tabular}
    
  \subsection{L'organisation}
    

\section{Présentation du projet}
    \subsection{Organisation}
    
    
    \begin{itemize}
      \item Ibrahima sur l'étude de GPG,
      \item Lucas sur l'attaque sur les KeyID,
      \item Les autres (Matthieu, Olivier et Pierre) sur le développement de l'application.
    \end{itemize}

  \subsection{Fonctionnalités}
    Les fonctionnalités ayant été complétées sont :\medbreak
  \begin{itemize}
  \item Exécution d'actions GPG \smallbreak
  Création, exportation, importations de clés (en local)\smallbreak
  \item Chiffrer / déchiffrer / signer / vérifier \smallbreak
  Chiffrer ou déchiffrer ou signer ou vérifier un fichier. \smallbreak
  \item Affichage des commandes, des retours et des erreurs \smallbreak
  L'utilisateur peut choisir d'afficher ou non les commandes, les retours et les erreurs associés à chaque action GPG. \smallbreak 
  \item Création / Modification / Suppression du profil utilisateur \smallbreak
  \item Modification de la toile de confiance \smallbreak
  Modification de la toile de confiance par un changement de niveau de confiance, l'ajout d'une nouvelle clé ou la modification d'une clé. \smallbreak
  \item Calcul d'une seconde pré-image pour une clé donnée \smallbreak
  Attaque permettant d'obtenir une nouvelle clé contenant le même KeyId que la première.
  \end{itemize}


  

  Les fonctionnalités qui ont été supprimées par rapport au plan initial sont :\medbreak
  \begin{itemize}
  \item Toile de confiance graphique\smallbreak
  Représentation de la toile de confiance sous forme de graphe\smallbreak
  \item Chiffrer / déchiffrer / signer / vérifier \smallbreak
  Chiffrer ou déchiffrer ou signer ou vérifier le texte contenu dans l'éditeur de l'interface.\smallbreak
  \item Edition de clé \smallbreak
  Quelques fonctions d'édition de clé ne sont pas implémentées, comme la révocation d'une clé ou la signature avec une clé en particulier \smallbreak 
  

  La partie interface graphique a été validée par le client, mais malheureusement l'attaque et l'étude sur GPG n'ont pas été validés car ils n'ont pas été rendus à temps.

  \end{itemize}

\section{Les problèmes rencontrés}

  Lors de la réalisation de ce projet nous avons dû faire face a
  différents problèmes aussi bien d'ordre organisationnel que technique,
  auxquels nous avons du remédier.

  \subsection{Problèmes organisationnels}

    Les difficultés organisationnelles rencontrées sont essentiellement
    survenues du fait de la modification de l'effectif de l'équipe durant la
    phase de réalisation du projet. 

    L'équipe initialement formée de sept développeurs,
    s'est trouvé réduite de deux personnes :
    \begin{itemize}
      \item Bertille Bouillie qui dès les deux premières semaines n'a plus donné
      aucune nouvelle, son abandon a seulement pu être officiellement pris en compte au bout d'un mois et demi.
      \item Guillaume Leroy qui a abandonné le projet au bout d'un mois et demi.
    \end{itemize}

    De plus Guillaume avant son départ définitif a changé de formation avec Lucas pour intégrer le Master GIL
    ce qui nous a posé des difficultés d'emploi du temps pour réaliser les différentes réunions quotidiennes
    de l'équipe pour faire avancer le projet.

    De plus les actions entreprises lors de l'avant projet n'ont pas suffi à l'ensemble de l'équipe 
    pour se former sur les différents outils utilisés lors de la phase de réalisation,
    puisque lors du premier sprint toutes les tâches attribuées a certains membres n'ont pas pu être
    réalisées suite à ces différentes lacunes.

    Nous avons, suite à cela pris du retard sur le premier sprint.

    Un plan d'action a alors été mis en place pour faire l'état des lieux des tâches
    faites, et des tâches réalisables avec le restant de l'équipe et le temps disponible.

    Ce plan a été présenté et discuté avec le client pour redéfinir le périmètre du projet.

    Suite à cela nous avons dû réattribuer les tâches à chacun pour pouvoir tenir la charge de travail,
    Lucas et Ibrahima ont souhaité se charger respectivement de l'attaque et de l'étude d'OpenPGP / GPG.

    Ainsi nous avons redéfini les tâches de développement de l'application
    entre Olivier, Matthieu (développeurs) et Pierre (Testeur).
    Les fonctionnalités de l'application ont également été redéfinies pour pouvoir être réalisables avec un effectif
    aussi réduit.

    De plus toute la réalisation de la visualisation de la toile de confiance a été réduite
    à l'affichage des niveaux de confiance d'une clé et non plus en dessin d'une toile de confiance graphique.

    Nous avons également à la fin du premier sprint suite à ces observations redéfini les rôles de chaque membres : 

    \begin{tabular}{|l|l|}
      \hline
      \bfseries{Noms}      & \bfseries{Rôles}                             \\
      \hline
      Olivier Thibault     & Chef de projet / Développeur                 \\
      Matthieu Fin         & Architecte / Chargé client / Développeur     \\
      Pierre Balmelle      & Testeur                                      \\
      Lucas Barbay         & Chargé de l'attaque                          \\
      Ibrahima Sorry Barry & Responsable de documentation OpenPGP / GnuPG \\
      \hline

    \end{tabular}

    
  \subsection{Problèmes techniques}

    \subsubsection{Maîtrise des outils}

      Lors de la phase d'avant projet nous avons organisé des sessions d’entraînement afin
      de prévenir du risque lié a l’incompréhension des outils utilisés tels que git, Qt, le langage C++,
      ou même le logiciel GnuPG.

      Seulement même si lors de l'avant projet l'ensemble de l'équipe semblait maîtriser ces
      différents outils nous nous sommes rendu compte lors de la phase de développement que ce n'était pas le cas
      pour tout le monde.

      Les outils utilisés ont été définis conjointement dès le 27 novembre, or dès le début de la phase de développement
      (c'est à dire au mois de janvier) certain membres n'avaient toujours pas installé les dits outils. Par conséquent
      ils ne pouvaient les maîtriser. Nous avons donc dû nous réunir pour installer et former a minima les différents membres
      sur ces outils, ce qui nous a fait perdre du temps sur le premier sprint de développement.

    \subsubsection{Maîtrise de GnuPG}
    
      Lors de la phase d'avant projet nous avons seulement étudier le logiciel gpg
      en superficie, ce qui nous a valu quelques surprises lors du développement.

      Notamment tout le mécanisme de gestion des entrées / sorties du logiciel n'as pas vraiment été
      étudié, or la base de l'application demandée repose sur la gestion de la communication avec
      gpg.
      En effet celui-ci n'écrit et ne lit pas toujours sur son entrée et sa sortie standard,
      en fonction des paramètres qu'on lui donne.

      Et quand il ne lit ou n'écrit pas sur ses entrée / sortie standard il lit et écrit sur un
      tty.

      Nous avons donc du trouver une solution à ce problème technique survenu lors du premier sprint,
      qui a consisté à développer une base de pseudo-terminal a l'aide de l'API POSIX, nous permettant
      d'exécuter un shell qui exécute gpg.
      Ainsi le shell s'occupe des descripteurs de fichier standard pour gpg et ce dernier peut accéder à un tty
      pour les commandes qui en nécessitent un.
      De plus notre pseudo terminal est composé d'une partie maître et d'un partie esclave,
      c'est la partie esclave qui s'occupe d'exécuter le shell avec gpg et la partie maître affiche
      sur sa sortie standard tout ce qui est écrit sur le terminal esclave, 
      et écrit sur le terminal esclave tout ce qui est écrit sur son entrée standard.

      Cela nous permet de regrouper tout ce qui est écrit et lu sur un même flux et donc de pouvoir
      complètement simuler l'exécution de commande gpg dans un terminal.
  

\section{Bilan du projet}
  
  \subsection{Compétences acquises}

    \subsubsection{Technique}
    
      Ce projet nous a beaucoup appris sur la partie technique du développement d'une interface graphique, en particulier l'utilisation du cadriciel (framework) Qt sous sa version 5, et de l'environnement de développement associé Qt Creator.

      Nous avons également gagné des compétences en C++, qui a été utilisé pour développer notre interface graphique.

      Les tests ont été l'occasion d'acquérir des connaissances sur le module Qt Test, et sur la manière d'effectuer des tests en général.

      Le projet a été l'occasion de découvrir GnuPG et ses nombreuses fonctionnalités.

    \subsubsection{Organisationnel}

      Nous avons retenu de ce projet qu'il est essentiel de s'assurer que l'équipe est effectivement formée sur les outils qui vont être utilisés ainsi que de faire l'inventaire de ces outils.
      Il faut aussi suivre régulièrement la progression de chaque membre dans la réalisation des tâches qui lui sont attribuées, pour être certain de l'état de progression de celles-ci et pouvoir s'adapter si elles ne sont pas finies à temps (ou finies plus tôt que prévu).




  \subsection{Axes d'amélioration}
  
    Le projet pourrait être amélioré en développant les commandes gpg prévues dans la STB initiale, mais qui ont dû être supprimées par manque de personnel, notamment pour l'édition de clé. 
    La toile de confiance graphique peut aussi être affichée sous forme de graphe, et offrir la possibilité de modifier une clé en effectuant un clic droit sur l'un des sommets du graphe (représentant une clé).
    La personnalisation de l'interface pourrait être développée, donnant par exemple l'opportunité de personnaliser les couleurs représentant la validité ou la confiance d'une clé, ou les informations affichées sur chaque clé.
    Le script de l'attaque peut également être amélioré. La recherche d'une nouvelle clef se fait en modifiant une clé générée et en comparant le haché de la modification jusqu'à ce qu'il corresponde à la clé ciblée. Selon la capacité de la machine, le script calcule entre 10 000 et 30 000 hachés par seconde, il faut donc plusieurs heures pour parcourir les $2^{32}$ possibilités. Les possibilités d'améliorations du script seraient donc de paralléliser les calculs ou d'utiliser la carte graphique pour effectuer plus de calculs.

  \subsection{Rétrospective}
	Si le projet était à refaire, il y a plusieurs choses que l'on ferait différemment. Nous commencerons par réaliser une étude poussée du logiciel GPG afin de comprendre chacune de ses fonctionnalités. Ensuite nous ferions le nécéssaire pour s'assurer que tous les membres de l'équipe ont une connaissance et une maîtrise suffisante des outils pour mener le projet à son terme.


\section{Conclusion}

	Pour conclure ce rapport, nous pouvons dire que ce projet nous a permis d'acquérir de nombreuses connaissances tel que l'utilisation poussée du logiciel GPG, le développement en C++ et l'utilisation du quadriciel Qt, l'utilisation des outils collaboratif tel que git et enfin des méthodes de gestion de projet qui nous aideront très prochainement dans le monde du travail.



\end{document}

