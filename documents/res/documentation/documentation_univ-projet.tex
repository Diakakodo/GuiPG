\documentclass{../univ-projet}

\usepackage[T1]{fontenc}
\usepackage[utf8]{inputenc}
\usepackage{DejaVuSans}

\logo{../logo_univ.png}
\author{Damien Picard}
\relecteur{Damien Picard}
\filiere{M1 SSI}
\projet{Univ-projet}
\projdesc{Une classe pour documenter ce projet,\\
une classe pour les documenter tous.}
\title{Documentation}
\version{1.0}

\histentry{1.0}{13/11/2013}{Première version}
\histentry{0.5}{10/12/2012}{Changelog pour la d\'emo}

\begin{document}
    \maketitle
    \tableofcontents
    \clearpage
    
    \section{Introduction} 
    \label{sec:intro}

        \subsection{Le but}
        \label{sub:le_but}
                Le but de cette classe est de permettre la production rapide et en
                {\LaTeX} de documents li\'es à la gestion de projets.

        \subsection{Pourquoi ne pas r\'eutiliser l'existant ?}
        \label{sub:pourquoi_ne_pas_r'eutiliser_l_existant_}
                Tout simplement car l'existant ne me convenait pas et que je le jugeais trop
            l\'eger, seule la gestion des fiches et des en-têtes/pieds de pages \'etaient pr\'esents.

                De plus de nombreux d\'efauts de conception \'etaient eux aussi pr\'esents, utilisation de
            valeurs non d\'efinie dans le fichier, utilisation de paquets non import\'es, bref
            c'est fait au pif et ça ça ne me convient pas.
    
        \subsection{Pourquoi celle ci serait elle mieux ?}
        \label{sub:pourquoi_celle_ci_serait_elle_mieux_}
                Parceque premièrement ce n'est plus un paquetage, mais une classe de document ce qui
            est moins ambigü à mon sens. En effet un paquetage ne devrait pas red\'efinir aussi 
            drastiquement la mise en page, seulement apporter des fonctionnalit\'es.

                Deuxièmement le plus grand soin a \'et\'e apport\'e a l'\'evitement des erreur
            de conception, tout ce dont la classe a besoin est import\'e dans la classe ou
            d\'efini dans la classe. Cela l'alourdit certes mais cela allège la source du
            document à produire.

                Finalement, c'est moi qui l'ai fait alors ça ne peut que être bien.

    \section{Comment ça fonctionne} 
    \label{sec:comment_a_fonctionne}
        
        \subsection{Principes g\'en\'eraux}
        \label{sub:principes_g'en'eraux}
                La majeur partie du comportement de cette classe est h\'erit\'e de la
            classe \verb?article?. Tout ce qui n'est pas mentionn\'e après est donc à consid\'erer
            comme dans la class \verb?article?.
        
        \subsection{Utilisation}
        \label{sub:utilisation}
                Pour utiliser cette classe c'est très simple il suffit de d\'eclarer votre
            document avec\\ \verb?\documentclass{<path-vers-univ-projet>}? sans l'extension \verb?.cls?.
        
        \subsubsection{Page de garde}
        \label{sub:page_de_garde}
                Bien que le comportement de article soit repris, il y a une page de garde
            entière, paramêtrable à l'aide des champs suivants (en plus des champs usuels):
            \begin{itemize}
                \item \verb?\logo?: Le chemin vers l'image servant de logo.
                \item \verb?\relecteur?: Le nom du relecteur.
                \item \verb?\filiere?: Le nom de votre filière.
                \item \verb?\projet?: Le nom du projet sur lequel porte le document.
                \item \verb?\version?: La version du document.
            \end{itemize}
            
                Il existe en plus une autre commande qui permet de paramêtrer le tableau de
            changelog affich\'e sur la page de garde:
            \begin{itemize}
                \item \verb?\histentry{version}{date}{description}?
            \end{itemize}
            Chacun des appel à cette macro rajoute une entr\'ee dans le tableau.
        
        \subsubsection{Les en-têtes}
        \label{sub:les_en_t_tes}
                Les en-têtes et les pieds de pages ont \'et\'es modifi\'es pour prendre
            en compte les diff\'erentes informations relatives au documents.
        
    \section{Évolutions possibles} 
    \label{sec:_volutions_possibles}
   
            À l'heure actuelle il manque certainement beaucoup de choses. Donc
        toute suggestion, ainsi que toute aide, sera la bienvenue.
\end{document}
