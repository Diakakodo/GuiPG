% !TEX encoding = UTF-8 Unicode
\documentclass{../res/univ-projet}

%Import des packages utilisés pour le document
\usepackage[utf8x]{inputenc}
\usepackage[francais]{babel}
\usepackage[T1]{fontenc}
%\usepackage{array}
%\usepackage{hyperref}
%\usepackage{tabularx, longtable}
%\usepackage[table]{xcolor}
%\usepackage{fancyhdr}
%\usepackage{lastpage}

\definecolor{gris}{rgb}{0.95, 0.95, 0.95}

%Redéfinition des marges
%\addtolength{\hoffset}{-2cm}
%\addtolength{\textwidth}{4cm}
\addtolength{\topmargin}{-1cm}
\addtolength{\textheight}{1cm}
\addtolength{\headsep}{0.8cm} 
\addtolength{\footskip}{-0.2cm}


%Import page de garde et structures pour la gestion de projet
%\usepackage{structures}

%Variables
\logo{../res/logo_univ.png}
\title{Terminologie et sigles utilisés}
\author{Bertille \bsc{Bouillie}}
\projet{GPG}
\projdesc{Interface Graphique GPG}
\filiere{M1SSI - Conduite de Projet}
\version{0.1}
\relecteur{Olivier \bsc{Thibault}}
\signataire{Magali \bsc{Bardet}}
\date{\today}

\histentry{0.1}{15/12/2014}{Version initiale.}

% -- Début du document -- %
\begin{document}

%Page de garde
\maketitle
\newpage
%La table des matiÚres
\tableofcontents
\newpage


\section{Glossaire}

\begin{description}
 \item[OpenPGP] : OpenPGP est un format de cryptographie de l'Internet Engineering Task Force (IETF), normalisé dans la RFC 4880.\\
 \item[Qt] : Qt est une API orientée objet et développée en C++ par Qt Development.\\
 \item[Toile de confiance] : représentera visuellement les liens entre les différents membres d'un réseau de clés.\\
\end{description}



\section{Abréviations}

\begin{description}
 \item[GnuPG] : Gnu Privacy Guard\\
 \item[GPA] : Gnu Privacy Assistant\\
 \item[KDE] : K Desktop Environment\\
 \item[KGpg] : KDE Graphical for privacy guard\\
 \item[OpenPGP] : Open Pretty Good Privacy\\
\end{description}



\section{Formalisme utilisé}

\begin{description}
 \item 
\end{description}



\section{Légendes et conventions de représentation}

\begin{description}
 \item 
\end{description}



\end{document}