\documentclass{../res/univ-projet}

%Import des packages utilisés pour le document
\usepackage[utf8x]{inputenc}
\usepackage[francais]{babel}
\usepackage[T1]{fontenc}
%\usepackage{array}
%\usepackage{hyperref}
%\usepackage{tabularx, longtable}
%\usepackage[table]{xcolor}
%\usepackage{fancyhdr}
%\usepackage{lastpage}

\definecolor{gris}{rgb}{0.95, 0.95, 0.95}

%Redéfinition des marges
%\addtolength{\hoffset}{-2cm}
%\addtolength{\textwidth}{4cm}
\addtolength{\topmargin}{-1cm}
\addtolength{\textheight}{1cm}
\addtolength{\headsep}{0.8cm} 
\addtolength{\footskip}{-0.2cm}


%Import page de garde et structures pour la gestion de projet
%\usepackage{structures}

%Variables
\logo{../res/logo_univ.png}
\title{Cahier des Recettes}
\author{Pierre \bsc{Balmelle}, Lucas \bsc{Barbay}}
\projet{GPG}
\projdesc{Interface Graphique GPG}
\filiere{M1SSI - Conduite de Projet}
\version{0.1}
\relecteur{Olivier \bsc{Thibault}}
\signataire{}
\date{\today}

\histentry{0.1}{31/10/2014}{Version initiale.}

% -- Début du document -- %
\begin{document}

%Page de garde
\maketitle
\newpage
%La table des matières
\tableofcontents
\newpage

\section{Introduction}

% Présentation succinte du sujet et hyp de travail.
Ce document est le Cahier de Recette (CDR) de la réalisation d'une interface graphique pour le logiciel GnuPG.
Réalisation d'une interface graphique permettant d'utiliser complètement l'outil OpenPGP
de façon intuitive et pédagogique.

\subsection{Fonctionnalités du logiciel}
\begin{itemize}
 \item Le logiciel est une interface graphique pour le logiel GnuPG.
\end{itemize}

\subsection{Liste des objets à tester}
\begin{itemize}
 \item L'interface graphique des fonctions GnuPG.
 \item La visualisation de la toile de confiance.
 \item L'implémentation de l'attaque sur les KeyID. 
\end{itemize}

\subsection{Contexte d'exécution des tests}
\begin{itemize}
 \item
\end{itemize}

\subsection{Choix technologiques et dispositions particulières}
\begin{itemize}
 \item 
\end{itemize}



\section{Documents applicables et de référence}
% Liste des
% - Références des documents quidefinissent formellement les principes
%   directeurs et le hypothèse de travail prise en compte pour l'établissement de la spécification.
% - Références des documents cités dans la STB au titre d'explication ou de justification.
Différents documents de référence :
\begin{itemize}
\item Définitions du standard OpenPGP \href{file:../../ressources/openPGP/rfc4880-en.pdf}{RFC 4880}
  et \href{file:../../ressources/openPGP/rfc2440-fr.pdf}{RFC 2440}
\item Le logiciel \href{https://www.gnupg.org/}{GnuPG} (GNU Privacy Guard) implantation Open Source
  de OpenPGP.
\item La \href{https://www.gnupg.org/gph/fr/manual.html#AEN541}{toile de confiance} de GnuPG
\item Editeurs graphiques existant 
  (\href{http://www.gnupg.org/related_software/frontends.en.html}{KGpg, GPA, Seahorse})
\item Exemple de visualisation d'une toile de confiance sur le site de 
  \href{https://www.archlinux.org/master-keys/#visualization}{archlinux}.
\end{itemize}

\section{Terminologie et sigles utilisés}
\textcolor{blue}{
  \begin{itemize}
  \item Glossaire ou dictionnaire
  \item Abréviations
  \item Formalisme utilisé
  \item Légendes et conventions de représentation
  \end{itemize}
}

\section{Environnement de test}
\subsection{Sites de réalisation des tests}


\subsection{Configurations matérielles utilisées}
\begin{itemize}
 \item PC n°1 : Debian avec Gnome
 \item i5-3210M @ 2.50GHz
 \item 6Go RAM

\end{itemize}

\subsection{Outils de tests mis en oeuvre}
\begin{itemize}
 \item 
\end{itemize}

\subsection{Jeux de données/Bases de données}
\begin{itemize}
 \item 
\end{itemize}

\subsection{Contraintes à prendre en compte}
\begin{itemize}
 \item 
\end{itemize}




\section{Responsabilités}

\begin{itemize}
 \item 
\end{itemize}


\section{Stratégies de tests}

\subsection{Description de l'approche et des phases de test}
\begin{itemize}
 \item 
\end{itemize}

\subsection{Campagne de test}
\begin{itemize}
 \item 
\end{itemize}

\subsection{Ordre d'exécution des tests}
\begin{itemize}
 \item 
\end{itemize}

\subsection{Critères d'arrêt des tests}
\begin{itemize}
 \item 
\end{itemize}


\section{Gestion des anomalies}



\section{Procédures de test}



\section{Jeux de données de test}

\section{Couverture de test}
\end{document}

